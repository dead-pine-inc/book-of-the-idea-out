\section{Заметки мёртвых математиков} % серия мини-статей, которую должная открыть извлечение корней из мтр 
\begin{epigraph}
    Победа сил разума над силами добра!
    \flushright{\normalfont Ильин Александр Александрович, 29 апреля 2016}
\end{epigraph}
\begin{center}
      \includegraphics[width=0.5\textwidth]{int_dp.pdf}
\end{center}

Рассмотрим простую задачу из 8 класса, но для матриц:
\begin{equation}
    X^2 = A.
\end{equation}
Сейчас мы будем рассматривать только такие матрицы \( A \), у которых все собственные различны. Прелесть таких матриц в том, что их нормированные собственные векторы определены единственным образом.

Выясним связь, между собственными векторами и значениями матриц X и A:
\begin{equation}
    X\vec{v}_i = \mu_i\vec{v}_i,\quad
    A\vec{v}_i = \mu_i X \vec{v}_i = \mu_i^2 \vec{v}_i = \lambda_i \vec{v}_i.
\end{equation}

Отсюда, собственные векторы матриц \( A \) и \( X \) совпадают, а собственные значения связаны соотношением
\begin{equation}
    \lambda_i = \mu_i^2.
\end{equation}

Теперь становится ясно, как построить матрицу \( X \):
\begin{enumerate}
    \item Привести матрицу A к диагональному виду.
    \item Извлечь корень из диагональных элементов.
    \item Сделать обратное преобразование.
\end{enumerate}

Или одной формулой
\begin{equation}
    X = T(\diag(\lambda_1,\ldots,\lambda_n))^\frac{1}{2}T^{-1}=T(T^{-1}AT)^\frac{1}{2}T^{-1},
\end{equation}
где матрица \( T \) составлена из собственных векторов матрицы \( A \).

Квадратный корень из числа имеет два значения, отличающиеся знаком, поэтому \( X \) может принимать \( 2^n \) различных значений (\( 2^{n-1} \), если одно из собственных значений нулевое).

Рассмотрим конкретный пример:
\begin{equation}
    A = \begin{pmatrix}
    1 & 8\\
    2 & 7
    \end{pmatrix}.
\end{equation}
Её характеристическое уравнение
\begin{equation}
    \lambda^2 - 8\lambda - 9 = 0,
\end{equation}
откуда собственные значения матрицы
\begin{equation}
    \lambda_1 = -1,\quad \lambda_1 = 9,
\end{equation}
а собственные векторы
\begin{equation}
    \vec{v}_1 = \begin{pmatrix}
    4 \\
    -1
    \end{pmatrix},\quad
    \vec{v}_2 = \begin{pmatrix}
    1 \\
    1
    \end{pmatrix}
\end{equation}
Проверим себя:
\begin{gather}
    T^{-1}AT =
    \begin{pmatrix}
        \frac{1}{5} & -\frac{1}{5} \\
        \frac{1}{5} & \frac{4}{5}
    \end{pmatrix}
    \begin{pmatrix}
        1 & 8\\
        2 & 7
    \end{pmatrix}
    \begin{pmatrix}
        4 & 1 \\
        -1 & 1
    \end{pmatrix} =\nonumber\\
    = \begin{pmatrix}
        \frac{1}{5} & -\frac{1}{5} \\
        \frac{1}{5} & \frac{4}{5}
    \end{pmatrix}
    \begin{pmatrix}
        -4 & 9\\
        1 & 9
    \end{pmatrix} =
    \begin{pmatrix}
        -1 & 0\\
        0 & 9
    \end{pmatrix}
\end{gather}
Матрица диагональная, собственные значения на своих местах, поэтоиу мы продолжаем. С точностью до знака, возможны 2 различных выбора матрицы \( X \):
\begin{equation}
    X_1 = T\begin{pmatrix}
        i & 0\\
        0 & 3
    \end{pmatrix}T^{-1} \text{ и }
    X_2 = T\begin{pmatrix}
        i & 0\\
        0 & -3
    \end{pmatrix}T^{-1}.
\end{equation}
В итоге, имеем
\begin{equation}
    X_1 = \begin{pmatrix}
        4 & 1 \\
        -1 & 1
    \end{pmatrix}
    \begin{pmatrix}
        i & 0\\
        0 & 3
    \end{pmatrix}
    \begin{pmatrix}
        \frac{1}{5} & -\frac{1}{5} \\
        \frac{1}{5} & \frac{4}{5}
    \end{pmatrix}= 
    \frac{1}{5}
    \begin{pmatrix}
        3+4i & 12-4i \\
        3-i & 12+i
    \end{pmatrix},
\end{equation}
\begin{equation}
    X_2 = \begin{pmatrix}
        4 & 1 \\
        -1 & 1
    \end{pmatrix}
    \begin{pmatrix}
        i & 0\\
        0 & -3
    \end{pmatrix}
    \begin{pmatrix}
        \frac{1}{5} & -\frac{1}{5} \\
        \frac{1}{5} & \frac{4}{5}
    \end{pmatrix} = 
    \frac{1}{5}
    \begin{pmatrix}
        -3+4i & -12-4i \\
        -3-i & -12+i
    \end{pmatrix}.
\end{equation}
Всего в данном случае имеется 4 корня: \( \pm X_1, \pm X_2 \).

В случае, когда хотя бы 2 собственных значения совпадают возможны два варианта. Если матрица диагонализуется (например, если это единичная матрица), то решений бесконечно много из-за произвольности выбора базиса собственных векторов. Случай, при котором матрица не приводится к диагональному виду мы предлагаем рассмотреть читателю \href{https://www.sharelatex.com/project/58285dc0dcf62380011eb003}{самостоятельно}.