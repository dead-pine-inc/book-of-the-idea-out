\section*{Пьеса}
\begin{center}
    { \Large Описание алгоритма написания книги }
    % Треш, угар и содомия

    Трагикомедия в трех актах.
\end{center}

\begin{epigraph}
    --- В нас начал пропадать дух олдфагости...\\
    --- Мы перестали писать в окна любимым женщинам...
    \flushright{\normalfont Вырвано из контекста разговора об ICQ, 14 октября 2016}
\end{epigraph}

Основана на реальных событиях с минимальным вкраплением художественного вымысла.\\

\begin{figure}[ht!]
    \centering
    \includegraphics[width=\textwidth]{reservoirdogs}
    \caption{Славным ублюдкам посвящается...}
\end{figure}

\begin{center}
    \Large Действие первое и единственное % ибо уникальное
\end{center}

\begin{center}
    \large Пролог
\end{center}

\vspace{1ex}

\note{Четверг. День, плавно перетекающий в вечер. Октябрь. 13 число. 2016 год.}

\note{Под осенним солнцем четверо джентельменов и дама обсуждают план мероприятий на пятничный вечер. Краткая выдержка из диалога.}

\begin{flushleft}
\vspace{1ex}

\dialog{Anthony}{Предлагаю игру: слова-поезда. Принцип прост: последняя буква слова = первой следующего. Примеры:
\begin{enumerate}
  \item нУ УдачИ И ИсполнениЯ ЯрыХ Хотений!
  \item Просто описать тебе ерунду, указав вариант теперешнего опуса.
\end{enumerate}}

\dialog{Лёха}{Если ихтиандр розового окраса -- атакуйте его!}

\dialog{Вова \remark{обращаясь к пришедшей Виктории, которую позвал Лёха}}{Оо, какие люди!}

\dialog{Вова}{Алекс, ты уверен, что у Вики достаточно устойчивая психика для этого круговорота безумия?}

\note[space]{Вика зловеще хихикает}

\dialog{Вова}{Хотя нет, мы же плоды её воображения, и она сама это подстроила.}

\dialog{Вика \remark{тщательно подбирая слова для игры Антона}}{Какое ехидное единоборство!}

\dialog{Лёха}{Вольдемар, скоро узнаем...}

\vspace{1ex}
\end{flushleft}

\begin{center}
    \large Акт 1
\end{center}

\vspace{2ex}

\note{Пятница. Вечер. Октябрь. 14 число. 2016 год.}

\note{Морось. Пожалуй это лучшее слово для описания погоды, которая шаталась по улице в ту ночь, постукивая горошинами капель в окна мирно дремавших горожан. 4 представительных джентельмена, сидя за круглым деревянным столом, режутся в преферанс и перекидываются мыслишками.}

\note{Действующие лица: Лёха, % а здесь можно написать что-то смешное
Anthony, % описывающее человека, как в обычных пьесах
Илья, % например,
Вова % -- старший брат младшего большого брата
}

\begin{flushleft}
\vspace{1ex}

\dialog{Лёха}{Anthony, я давеча нашёл прелюбопытнейшую картинку для нашей книги. \href{http://cs8.pikabu.ru/post_img/2016/10/14/5/1476431396146890414.jpg}{Не желаете взглянуть?}}

\dialog{Anthony}{Пытаюсь как-нибудь привязать шутку про Билли Милигана к этой картинке: <<И тут я осмотрел все свои личности>> (что то в этом духе -- надо додумать)}

\dialog{Лёха}{Что-то типа: <<Я сижу один, а меня окружают одни дураки>>?}

\dialog{Anthony}{Можно и сторону одиночной камеры заключения обыграть.}

\dialog{Лёха \remark{немного повременив}}{Написание хорошего чёрного юмора -- сложная задача.}

\dialog{Anthony}{Вообще качественной более-менее приличной шутки.}

\dialog{Лёха}{Ну в чёрном нужно ещё суметь не оскорбить всяких идиотов.}

\dialog{Лёха \remark{загадочно улыбаясь}}{Или наоборот...}

\dialog{Илья \remark{смеясь}}{Тут скорее наоборот.}

\dialog{Anthony}{Точнее оскорбить так, чтобы они этого не поняли.}

\dialog{Илья}{<<Множественные дураки Билли Миллигана>>. Документальный роман Дэниела Киза}

\dialog{Лёха}{<<Я и мои дураки>>. Автобиография Билли Милигана}

\dialog{Илья}{Дартаньян и три дурака}

\dialog{Лёха}{Хорошо, что дураки, а не \ldots}

\note[space]{Илья смеётся}

\dialog{Anthony \remark{комментирует сказанное}}{Оправдательная речь директора автоваза: <<Хорошо, что дураки, а не \ldots>>}

\dialog{Лёха}{А не брак (здесь должна быть отсылка к переписке в вк)} % вставить ссылку на соответствующую главу в книге

\dialog{Anthony}{Предлагаешь остальным её показать?} % в оригинале: Предлагаешь её сюда завернуть?

\dialog{Лёха}{Да}

\note[space]{Anthony достаёт из рукава жестом волшебника какую-то мятую бумажку сомнительной свежести, разворачивает так, чтобы все видели текст и начинает читать}

\begin{miamalist}
  \item[Anthony:] У меня такой вопрос: почему слово брак означает свадьбу и негодную деталь?

  Может его стоит отнести к разряду нецензурных? Т.к. только мат люди используют и для радости, и для описания печали)

  \item[Лёха:] Придётся тогда писать бр*к

  И все сразу -- что это за слово???!

  \item[Anthony:] бра*

  б*ак

  \item[Лёха:] б**к
  \item[Anthony:] быык)
  \item[Лёха:] беек
  \item[Anthony:] блик
  \item[Лёха:] блок
  \item[Anthony:] баюк (котёнок Баюна)
  \item[Лёха:] и самое неочевидное -- б-звезда-звезда-к
  \item[Anthony:] одним словом: хорошую вещб браком не назовут)
  
  так говорят они
  
  в смысле, пословицы
  
  хотя и люди тоже подойдут
  
  2-х смысленно: хотя и люди тоже подойдут
  \item[Лёха:] брак всему голова
  
  семь раз отмерь, один раз брак
  \item[Anthony:] сделал брак, иди переделывай
  \item[Anthony:] не откладывай на завтра то, что можешь сделать браком сегодня
  \item[Лёха:] брак на брак не приходится
  \item[Anthony:] Всемогущ бог, да хитёр брак
  \item[Лёха:] Сколько человека не корми, всё равно на брак смотрит
  \item[Anthony:] работа не брак, сама себя не сделает
  \item[Лёха:] брак браком вышибают
  \item[Anthony:] без труда не сделаешь брака ни черта
  \item[Лёха:] нет так страшен брак, как его малюют
  \item[Anthony:] новая подподрубрика?
  
  бракованные идеи
  \item[Лёха:] поговорки на новый лад?
  \item[Anthony:] на новый брак
  \item[Лёха:] точно!
  \item[Anthony:] сейчас смотрю на нашу переписку и окончательно сформулировал и без того витавшую в воздухе гипотезу
  
  всё начинается от простейшего и эволюционирует к тому, что мы в конце будем называть идеалом -- нет такого, что вдруг -- бац -- и нате! готово совершенство
  
  закон Вселенной
  
  шах и мат
  \item[Лёха:] шах, мат и нате
  \item[Anthony:] НАТЕ -- для двупрочтения)
  \item[Лёха:] здесь в любом смысле хорошо звучит
\end{miamalist}

\note[space]{В воздухе повисла минутная тишина}

\dialog{Илья \remark{пытаясь продолжить беседу}}{не говори брак, пока не поломаешь\\
не всё коту качество, будет и брак\\
брак с возу -- конвейеру легче\\
в ногах брака нет\\
глядит в книгу, видит брак}

\dialog{Anthony}{брак с ленты -- конвейеру легче -- что-то вроде осовременивания пословиц}

\dialog{Anthony}{в ногах \textbf{Болта} брака нет}

\dialog{Лёха \remark{вставая из-за стола}}{Они же на новый лад/брак}

\dialog{Anthony \remark{улыбаясь и отслеживая передвижения Лёхи}}{Ну да, ну да}

\dialog{Лёха \remark{взяв с полки книгу}}{Спонсор следующей фразы --- \href{http://www.mista.ru/pogovorki.htm}{сборник пословиц и поговорок}:

\begin{itemize}
  \item[] Без брака бракованные.
  \item[] Брак в помощь.
  \item[] Брак создал, брак и забрал.
  \item[] Брак всё стерпит.
  \item[] Была у собаки хата, брак пришел -- она сгорела.
  \item[] В ногах брака нет.
  \item[] Там хорошо, где брака нет.
  \item[] Вот где брак зарыт.
  \item[] Вывести брак на чистую воду.
  \item[] Где браки зимуют.
  \item[] Брак -- не тётка.
  \item[] Два сапога -- пара, а три -- брак.
  \item[] Дело пахнет браком.
  \item[] Брак познаётся в беде.
  \item[] Брака не хватает.
  \item[] И швец, и жнец и бракоделец.
\end{itemize}}

\dialog{Anthony \remark{сдавая карты}}{А указывать источники как спонсоров --- интересная задумка}

\note[space]{Лёха кладет книгу, возвращается на место}

\dialog{Anthony \remark{после секундных раздумий продолжает}}{Надо будет её в книжку привить\ldots}

\dialog{Лёха}{Добрый доктор Антон сделает прививку и вашей книге}

\dialog{Илья}{Книжный грипп?}

\dialog{Лёха \remark{\href{http://i5.imageban.ru/out/2014/09/04/442aff271469c9b3f514584819fcc35c.jpg}{изображает лицом доктора Хауса}}}{Возможно, а то может быть и книжчанка}

\dialog{Илья}{dr. Book-us}

\dialog{Илья}{Хм, <<Во все книжные>>}

\dialog{Илья}{Теория большой книги}

\dialog{Лёха}{11 литературных друзей}

\dialog{Илья \remark{шёпотом}}{книжки, кофе}

\dialog{Илья \remark{громче, показывая на стол}}{2 стола?}

\dialog{Лёха \remark{задумываясь}}{шкафа, бобра, кота\ldots}

\dialog{Илья \remark{показывает большой палец вверх}}{Книжки, кофе, 2 кота!}

\note[space]{Вова возвращается в комнату с кружкой свежеразлитого бренди}

\dialog{Вова}{Брак без водки -- деньги на ветер!}

\dialog{Вова \remark{выпивая залпом весь напиток, продолжает, слегка морщась}}{Семь раз отмерь, один раз брак. Брак браку брак уже было?}

\dialog{Anthony}{Так можно всё что угодно переделать:\\Было у отца 3 сына.\\Старший был умён,\\средний силён,\\а ещё один -- бракован.}

\dialog{Вова}{Было у отца три сына:\\
Старший умный был детина,\\
Средний был и так, и сяк,\\
Младший -- откровенный брак!}

\dialog{Вова \remark{блаженно улыбаясь}}{Книжки, кофе, 2 кота! -- Это просто милота}

\note[space]{Некоторое время ничего не слышно, кроме тихого шелеста карт}

\dialog{Anthony}{Да уж пятница определённо вышла плодотворной на идеи --- попрежнему жду ваших иллюстраций с завтрашней философии}

\dialog{Лёха \remark{уклончиво}}{Всё зависит от музы\ldots}

\dialog{Anthony}{От скучности лекции}

\dialog{Вова \remark{хитро прищурившись}}{Антон, ты опять во времени запутался -- завтра ждать надо, а не по-прежнему}

\dialog{Вова}{Future Simple вместо Present Continiuos надо}

\dialog{Anthony}{Хочется ответить цитатой современного мёртвого/живого поэта/рэпера (шрёдингера?) --- сегодня завтра станет вчера}

\dialog{Вова}{С каких пор мэр Киева -- репер?}

\dialog{Anthony}{Это гуф -- забыл Лену?}

\dialog{Вова \remark{улыбается}}{Я такие вещи не слушаю}

\dialog{Илья}{Кличко = Лена?}

\dialog{Лёха \remark{не совсем ясно о чём}}{Мёртвая вещь}

\dialog{Anthony}{Очень нерекомендую) особенно перед завтраком}

\dialog{Лёха}{Чтобы завтра не стало сегодня!\\
Ну или не только завтра.}

\dialog{Вова \remark{разговаривая разными голосами}}{-- Не слушайте перед завтраком русский рэп.\\
— Так, помилуйте, другого-то низкосортного говна и нет!\\
— Вот никакое и не слушайте!}

\note[space]{Снова повисла тишина. Илья уходит на кухню за порцией бренди, Anthony сидит со скучающим видом}

%\dialog{Anthony}{Мне кажется или стикеры --- это какая то нездоровая тема?}
\dialog{Anthony \remark{в тоске смотря в окно}}{Мне кажется или пантомима --- это какая то нездоровая тема?}

\note[space]{Лёха пожимает плечами, Вова изображает Фрейда}

\dialog{Лёха}{Ну не на столько же!}

%\dialog{Anthony}{Вывод вечера --- беседа переходит в угар, когда в ход идут стикеры.}
\dialog{Anthony}{Вывод вечера --- беседа переходит в угар, когда в ход идёт пантомима}

%\dialog{Лёха}{Вывод: не нужно нюхать стикеры!}
\dialog{Лёха}{Вывод: не нужно нюхать краску для мимов!}

%\dialog{Anthony}{у нас была беседа в телеграмме. несколько идей и стикеры, но мы боялись их трогать...}
\dialog{Anthony}{у нас было выступление на аллее, несколько номеров и пантомим, но мы боялись их изображать...}

%\dialog{Лёха}{и куча различных смайлов всех цветов и расцветок}
\dialog{Лёха}{и куча различных свистелок всех цветов и расцветок}

%\dialog{Anthony}{а ещё кто-то постоянно пересылал сообщения из вк}
\dialog{Anthony}{а ещё кто-то постоянно пересказывал события за прошедший день}

\dialog{Anthony}{самокритичный ублюдок)}

\dialog{Вова}{У меня алиби.}

\dialog{Лёха}{А я в домике.}

\dialog{Вова}{<<самокритичный ублюдок)>>}

\dialog{Лёха}{юзай картинки}

\dialog{Вова}{долго, дорого, нахуz не нужно}

\dialog{Вова}{Предлагаю разместить эту цитату в коментариях в коде нашего форума}

\note[space]{Вика врывается в комнату и всплёскивает руками}

\dialog{Вика}{Вова матом ругается!}

\dialog{Лёха \remark{делая вид, что не заметил Вику, говорит тоном меланхоличного флегматика в пустоту}}{Может быть ещё ASCII артов и на каждой странице?}

\dialog{Вова}{Вова так разговаривал каждым летом, когда во дворе бегал)}

\dialog{Вова \remark{подмигивая Вике}}{Это красный, детка!}

\dialog{Anthony \remark{тоном диванного эксперта}}{Вова не ругается, а ясно формулирует свои эмоции в словестных выражениях определённой направленности}

\dialog{Anthony \remark{полушёпотом добавляет}}{направленность снова 2смысленна}

\dialog{Вика \remark{тоном учителя младших классов}}{Определенной нравственности}

\note[space]{Вика уходит из комнаты с лицом, полным обреченности, по дороге чуть не сбивает Илью, вернувшегося с кружкой. Немного бренди расплёскивается на пол.}

\dialog{Anthony}{К чёрту нравственность --- Только водоворот безумия --- только bookcore!}

\vspace{1ex}
\end{flushleft}

\begin{center}
    \large Part 2
\end{center}

\vspace{2ex}

\note{Пятница. Вечер плавно перетекает в ночь. 14 число. Октябрь. 2016 год.}

\note{Дождь за окном прекратился, оставив за собой на улице мелкие лужицы. Из подъездов и пивнушек стали вылазить бродяги, пьянчужки и ночные бабочки.}

\note{В комнате с круглым деревянным столом для преферанса царит явное оживление.}

\note{К действующим лицам добавляется Кот.}

\begin{flushleft}
\vspace{1ex}

\dialog{Лёха \remark{оживлённо}}{Анархия?}

\dialog{Вова}{Мать порядка?}

\dialog{Anthony}{А разве у нас не она?}

\dialog{Вова \remark{с ехидной улыбкой}}{Напиши, как за тобой приедут.}

\dialog{Anthony \remark{театрально закатывая глаза}}{Чёрный воронок уже вылетел!}

\dialog{Лёха}{Я могу выехать.}

\dialog{Anthony \remark{продолжает язвить}}{Сушите сухарики --- пишите мелким почерком.}

\dialog{Илья \remark{стоя у окна и поглаживая Кота}}{Переписка голубями?}

\dialog{Илья \remark{резко обернувшись, смотрит на Лёху}}{и Голубевыми?}

\dialog{Вова \remark{также смотря на Лёху}}{Алекс, меня подбери по пути.}

\dialog{Вова \remark{кивая в сторону Ильи с кружкой}}{и коньячок тоже)}

\dialog{Anthony}{Такси Лёха --- <<Я могу выехать>>}

\dialog{Вова \remark{смеясь, встаёт из-за стола}}{<<А могу не выехать>>. Троллинг Шрёдингера.}

\dialog{Илья}{<<А могу такси вам вызвать!>>}

\dialog{Вова \remark{приплясывая гопак}}{<<А могу ментов, диктуйте адрес!>>}

\dialog{Илья \remark{напевая}}{Мой адрес сегодня такой:}

\dialog{Лёха \remark{не попадая в такт}}{Пр-кт Ленина, 28, Волгоград}

\dialog{Anthony \remark{бросает отрешённый взгляд в пространство}}{\href{http://pine-forum.herokuapp.com/}{не дом и не улица}}

\dialog{Вова}{а кафедра философии и права}

\dialog{Илья}{вызывать ментов на лекцию?}

\dialog{Лёха}{Чтобы не было скучно!}

\dialog{Anthony \remark{всё так же отрешенно}}{Пр-кт Университетский, 100 --- Кафедра английского языка --- на следующей остановке загляните}

\dialog{Anthony \remark{собрался с мыслями}}{через неделю.}

\note[space]{Илья изображает гангстера}

\dialog{Илья}{Всем лежать, никому не двигаться!}

\note[space]{Вова изображает ваххабита}

\dialog{Илья \remark{вспоминая фразу Anthony про английский язык}}{эээ, донт мув, донт мув}

\note[space]{Лёха изображает Че, который лихо заливается дьявольским смехом, пробирающим до самых костей жалкую плоть смертных людишек. Anthony смеётся под свой не в меру длинный и горбатый нос}

%\dialog{Anthony}{Снова стикеры --- я сваливаю!}
\dialog{Anthony}{Снова пантомима --- я сваливаю!}

\dialog{Илья \remark{с упрёком}}{Чё ты как не мужик-то?}

\note[space]{Илья тычет в Anthony Котом}

\dialog{Вова}{Нормально ж начинали.}

\dialog{Лёха \remark{пародирует голос Anthony}}{I don't live this planet anymore}

\dialog{Anthony}{Быть мужиком: 200.000 лет назад --- убить мамонта. 21 век --- выдержать стикер-атаку.}

\dialog{Илья \remark{с арабским акцентом}}{стикер-бомбинг!}

\dialog{Anthony \remark{задумчиво}}{Некстати говоря}

\dialog{Лёха \remark{немного язвительно}}{Стикеры подгорают?}

\dialog{Anthony \remark{не замечая этого}}{может составить этакий список мужикости по временам или столетиям?}

\dialog{Anthony}{19 век -- быть подстреленным на дуэли и выжить!}

\dialog{Anthony \remark{машет рукой и выкидывает пятюню}}{Привет, Галуа!}

\dialog{Anthony \remark{бубнит}}{минутка черного юмора}

\dialog{Илья}{<<Пушкин, чё ты дохнешь, чё не мужик?>>}

\dialog{Лёха}{Привет от Галуа \remark{\href{http://i031.radikal.ru/1404/5d/e6a261c33899.gif}{изображает крутящегося в гробу Галуа}}}

\dialog{Anthony \remark{с радостью во вновь загоревшихся глазах}}{Наркомания от Лёхи --- всё норм --- я остаюсь)}

%\dialog{Вова \remark{тоном морпеха}}{Задавим их стикерами!}
\dialog{Вова \remark{тоном морпеха}}{Задавим их!}

\dialog{Лёха \remark{тоном нашкодившего Карлсона}}{Мне просто завезли свежего ...}

\note[space]{Все смотрят на Лёху в ожидании конца фразы}

\dialog{Илья \remark{не стерпев}}{Медведя?}

\dialog{Вова \remark{подхватывая}}{Галуа?}

\dialog{Лёха}{Подойдёт любой вариант}

\dialog{Лёха \remark{пародируя голос из рации}}{радирую важную информацию: мы ушли с маршрута}

\dialog{Anthony}{Пилот, где/куда мы маршрутировали?}

\dialog{Лёха \remark{тоном алкоголика после глубокого запоя}}{мы на дне}

\dialog{Илья}{route 60?}

\dialog{Илья}{на острове}

\dialog{Лёха}{Где Джек?}

\dialog{Илья}{за водой пошел}

\dialog{Илья}{мне больше интересно}

\dialog{Илья}{где Харли?}

\note[space]{Вова уходит из комнаты, Кот бежит за ним}

\dialog{Илья \remark{вдогонку Вове}}{и торчок}

\dialog{Anthony \remark{выпав из коматозной задумчивости}}{что за ? о чём вы --- я потерялся) Человек за бортом!}

\dialog{Илья \remark{смеясь}}{Тут сотни людей за бортом.}

\dialog{Лёха}{Кидай ему пакет с коксом!}

\dialog{Илья \remark{делает замах рукой}}{Пусть цифры пишет.}

\dialog{Anthony \remark{шмыгая странно носом}}{так может корабля то и нет, а капитан то голый)}

\dialog{Илья \remark{ничуть не смутившись}}{до костей!}

\dialog{Anthony \remark{оглядел комнату}}{А Вова уже выехал!}

\dialog{Anthony}{Упрямый ублюдок!}

\dialog{Илья}{На философию?}

\dialog{Anthony}{ваши прилагательные на бочку}

\dialog{Anthony}{на философскую бочку}

\dialog{Илья \remark{одновременно с Anthony}}{сделай бочку!}

\dialog{Илья \remark{смутившись}}{философское -- уже прилагательное}

\dialog{Илья \remark{подняв глаза, настойчиво}}{т.ч. сделай философскую бочку!}

\dialog{Anthony \remark{пытается показать бочку на рисунке}}{\href{https://thumbs.dreamstime.com/thumb_850/8505013.jpg}{Рисунок}}

\dialog{Лёха}{А где место для человека?}

\dialog{Anthony \remark{поясняет содержимое бочки}}{содержит этанол --- всё что нужно для философии}

\dialog{Anthony \remark{в ответ на вопрос Лёхи}}{внутри}

\dialog{Лёха \remark{поднимает густые брови}}{а дверь тогда где?}

\dialog{Лёха}{только не говори, что внутри}

\dialog{Anthony \remark{голосом наркомана}}{это загадочная философская бочка}

\dialog{Лёха \remark{поднимает руки к небу и выдаёт}}{\href{http://sad.co.ua/wp-content/uploads/2014/07/dveri-bochka.png}{спасибо, интернет}}

\dialog{Anthony \remark{убито и весело одновременно}}{дверей нет --- границ тоже --- всё дзен}

\dialog{Anthony \remark{глядя на часы, которые показывают 22:23}}{22:22}

\dialog{Anthony}{блин}

\dialog{Anthony \remark{опять бубнит}}{через сутки надо повторить}

\dialog{Anthony \remark{оживленно}}{или --- вперёд на запад!}

\dialog{Лёха \remark{изображая походку ковбоя}}{На дикий запад!}

\note[space]{Пару минут спустя в комнату забегает Кот, прыгает к Лёхе на колени. В дверном проеме появляется Вова, вернувшийся из ванной}

\dialog{Вова \remark{радостно}}{Торчок вернулся!}

\dialog{Лёха \remark{глядя на Вольдемара}}{закинулся и вернулся?}

%\dialog{Вова}{ага, я снова вижу стикеры}
\dialog{Вова}{ага, я снова вижу призраков}

\dialog{Anthony}{хороший мет}

\dialog{Вова}{я не упрямый, я больной}

\dialog{Вова \remark{гундося}}{чёртов насморк}

\vspace{1ex}
\end{flushleft}

\begin{center}
    \large Part 3
\end{center}

\vspace{2ex}

\note{Пятница. Ночь. Октябрь. 14 число. 2016 год.}

\note{За окном тускло горит уличный фонарь. С деревьев падают листья, покрывая одеялом теперь уже спящих на скамейках пьянчуг. Небо, ранее заволоченное тучами, медленно проясняется, открывая взору практически полную Луну.}

\note{За столом осталось трое игроков, занятые партией в дурака. Четвертый джентельмен в комнате мирно подрёмывает на стуле с Котом на коленях.}

\note{Действующие лица: те же личности.}

\begin{flushleft}
\vspace{1ex}

\dialog{Вова \remark{просыпаясь}}{Люди, вы где? Чего затихли?}

\note[space]{Кот в панике убегает от Вовы к Лёхе на колени}

\dialog{Лёха}{просто нечего сказать по этому поводу}

\dialog{Вова \remark{потирая глаза}}{Закончился юмор в юморницах?}

\dialog{Anthony}{кокс выветривается -- батареи не греют -- мы трезвеем)}

\dialog{Илья}{бракованный юмор}

\dialog{Вова}{А у меня греют}

\dialog{Anthony \remark{с лёгкой завистью}}{везучий ублюдок)}

\dialog{Лёха с не совсем лёгкой завистью}{присоединяюсь к выше сказанному}

\dialog{Илья}{у нас тоже греют}

\dialog{Anthony}{в прнципе этот термин превращает любую фразу в тарантиновскую}

\dialog{Вова протяжно завывает}{Ворошиловский район, ветер северный}

\dialog{Anthony подхватывает}{дует из окна -- зла немерено)}

\dialog{Лёха}{Окна затвори и зло угомони}

\dialog{Anthony}{щели замажь -- печку / баньку истопи}

\dialog{Вова}{Пирог испеки, соседей накорми}

\dialog{Лёха}{Собрались хозяюшки!}

\dialog{Вова}{Студень замути, немного накати}

\dialog{Anthony c улыбкой}{отчаянные домохозяины}

\dialog{Вова, критично}{Не ну так себе затея для сериала}

\dialog{Лёха, задумчиво}{Смотрю на какую аудиторию}

\dialog{Anthony, истошно пародируя истеричку}{Я выращимаю мандарины в Волгограде, Карл, a ты говоришь так себе идея?}

\dialog{Лёха}{Мандариновый магнат Антон}

\dialog{Anthony}{Антонио --- немного средиземноморья}

\dialog{Вова}{Ты ещё скажи, что мандарин не твой, тебе подкинули}

\dialog{Антонио}{так и было --- он сам пришёл}

\dialog{Лёха}{Вкусный, спелый, но не мой.}

\dialog{Вова}{немой в одно слово выглядит лучше}

\dialog{Антонио}{немой в любом виде выглядит немного загадочно}

\dialog{Лёха}{это дело вкуса}

\textbf{Вова снова изображает Фрейда}

\dialog{Лёха с лёгкой иронией}{Спасибо доктор, но нам не нужна консультация.}

\dialog{Антонио прикрывает ладонью лицо, мурлыкая себе под орлиный носоклюв}{опять рецидив}

\dialog{Илья}{Опять, вы серьезно?}

\dialog{Вова}{*стикер с корейцем*}

\dialog{Антонио}{Где старые добрые олдскульные словестные аськи?}

\dialog{Илья}{О-оу}

\dialog{Илья}{Ржунимагу}

\dialog{Вова}{*ROFL*}

\dialog{Илья}{B)}

\dialog{Антонио ободрённо}{вот оно!}

\textbf{Лёха начинает тыкать Котом в Илью}

\textbf{Илья забирает у него Кота и начинает им тыкать в Вольдемара}

\textbf{Лёха отбирает Кота и зачем то тычет им в монитор выключенного компьютера}

\dialog{Вова \remark{со столетней тоской в голосе и обречённым взглядом приговорённого к расстрелу}}{такой олдскул намечался, а вы опять всё засрали котиками}

\dialog{Илья}{*FACEPALM*}

\dialog{Вова}{\href{https://icq.com/windows/ru}{скатилась асечка...}}

\dialog{Лёха \remark{бодрым тоном могильщика}}{Закапывайте}

\dialog{Antonio \remark{изображая Ипполита}}{в нас начал пропадать дух олдфагости...}

\dialog{Илья}{так мейл же}

\dialog{Вова \remark{намыливая шапку}}{мы перестали писать в окна любимым женщинам....}

\dialog{Antonio}{фраза вечера) недели!}

\textbf{Илья корчится от заздирающего его смеха}

\dialog{Лёха}{2х смысленная}

\dialog{Илья}{Все зависит от ударения}

\dialog{Antonio}{Всё зависит от места удара (ударения)}

\dialog{Илья}{Чак норрис ставит ударение на один и тот же слог?}

\dialog{Илья}{Мисье француз}

\dialog{Вова}{франсуа}

\dialog{Вова}{Мой брат смотрит на ютубе ролики с поняшками в озвучках на различных языках. А как проходит вечер у вас?}

\dialog{Лёха}{Я тут с какими-то людьми в преферанс зависаю.}

\dialog{Илья}{Слушаем, как твой брат смотрит на ютубе ролики с поняшками в озвучках на различных языках}

\dialog{Antonio}{Господа, присяжные заседатели, просто прохожие и ежи с ними --- для добивания страниц и просто для памяти --- предлагаю запилить этот вечер в книжку --- под рубрикой пятничный угар}

\textbf{Мысль мелькнула в голове Антона}

\dialog{Antonio}{название надо для рубрики перепридумать}

\dialog{Илья}{Пятница, 14е}

\dialog{Вова}{Пятничное моё}

\dialog{Antonio}{ещё}

\dialog{Лёха}{Пятницкое}

\dialog{Илья \remark{с контонским акцентом}}{5низза?}

\dialog{Вова}{5низзя? 5 ниндзя?}

\dialog{Лёха}{Пятничные посиделки|полежанки|пописанки|...}

\dialog{Лёха}{насчёт 5 ниндзя -- формально участвуют только 4}

\dialog{Илья}{Пописанки}

\dialog{Вова}{написанки}

\dialog{Лёха}{насиделки и належанки}

\dialog{Вова \remark{отмахивается от летучих мышей}}{Страх и ненависть в пятницу}

\dialog{Вова}{Назови это послесловием. Или предисловием.}

\dialog{Antonio}{вместословием}

\dialog{Вова}{или приложение А: описание алгоритма написания книги}

\dialog{Antonio}{@citrux это интересно}

\dialog{Вова}{@citrux -- это не только 56 килограмм диетического мяса, но и всегда интересно}

\dialog{Antonio}{но надо отметить в подназвании пятницу}

\dialog{Вова}{у тебя 7 пятниц на неделе}

\dialog{Лёха}{хорошая наверное неделя}

\dialog{Вова}{алгоритм -- инвариант преобразования сдвига по времени}

\dialog{Вова}{ты инициируешь чат случайным трешем из контакта и понеслась}

\dialog{Antonio}{не случайным, а тщательно подобранной наркоманией}

\dialog{Antonio}{мы тут вам не эти}

\dialog{Вова}{Тема диссертации: изучение устойчивости динамической системы с генерацией юмора}

\dialog{Лёха}{Эмиссия юмора в вакууме}

\dialog{Вова}{терморектальная)}

\dialog{Вова}{кстати, эмиссия юмора в вакууме может происходить только в виде пантомимы)}

\dialog{Лёха}{Кто и когда это доказал?}

\dialog{Вова}{Я только что}

\dialog{Antonio}{Гипотеза Абрахманова за нумером 97}

\dialog{Вова}{а потом космонавты 200 лет будут выяснять -- так это или не так}

\dialog{Antonio}{не имеется желания замутить 100 абдрагипотез?}

\dialog{Вова}{абдрипотез}

\dialog{Вова}{ну не знаю, они ж качественные должны быть}

\dialog{Вова}{покачественнее идей для стартапа}

\dialog{Antonio}{серьёзно? тебя не граничивают рамки. можно не сто}

\dialog{Antonio изображает Фрейда}{69}

\dialog{Вова}{не, максимум 13}

\dialog{Antonio}{давай}

\dialog{Вова}{при этом не будет 4, 9 и 13}

\dialog{Antonio}{зарубим отдельную главу. Или даже книгу в книге в книге...}

\dialog{Вова}{А первая гипотеза -- Гипотез 4, 9 и 13 не существует}

\dialog{Antonio}{но количественно их всё = будет 13?}

\dialog{Вова}{точнее, гипотез 4, 9  и d не существует}

\dialog{Antonio}{так лучше}

\dialog{Вова}{ну в итоге их 10 останется}

\dialog{Antonio}{@FreeCX вернулся}

\dialog{Лёха голосом Фрейкенбок}{туточки я}

\dialog{Antonio \remark{задумчиво}}{хотя восстал звучит пафосней}

\dialog{Antonio \remark{к Вове}}{цитрусовый ублюдок, ещё 9}

\dialog{Вова}{бесславные ублюдки}

\dialog{Antonio}{безцельный}

\dialog{Лёха}{пока бесславные)}

\dialog{Antonio}{славные ублюдки}

\dialog{Antonio}{милоты немного}

\dialog{Лёха}{опять котиков?}

\dialog{Antonio}{нее}

\dialog{Вова}{ну можно ещё как в бешеных псах -- я мистер оранжевый, Антон -- розовый, Алекс -- зелёный}

\dialog{Antonio}{почему я розовый?}

\dialog{Лёха}{а я не смотрел}

\dialog{Antonio}{был же чёрным на форуме?}

\dialog{Вова}{А фон?)}

\dialog{Вова}{кстати, Тони, ты процитировал фразу из фильма}

\dialog{Antonio}{необъективно}

\dialog{Antonio}{и зачем ограничиваться видимым диапазоном? мы, блин, физики или где?}

\dialog{Вова поёт в стиле Буратино}{УЛЬ}

\dialog{Вова}{ТРА}

\dialog{Вова}{ФИО}

\dialog{Вова}{ЛЕТ!!!}

\dialog{Antonio}{456 нм}

\dialog{Вова}{Ты хвастаешься?}

\dialog{Antonio}{Bовчика сегодня можно на цитаты записывать)}

\dialog{Вова}{У меня сегодня бенефис)}

\dialog{Лёха}{ладно я мухожук. И до завтра}

\dialog{Вова}{Давай, Владимирыч, до завтра}

\dialog{Вова}{а я только разогрелся...}

\dialog{Antonio}{итак они сошлись -- вода и пламень. лёд и камень)}

\dialog{Antonio}{анто и вова}

\dialog{Вова}{эт хреново}

\dialog{Вова}{Ладно, завтра увидимся}

\dialog{Antonio}{Последний герой!}

\dialog{Вова}{Может я на философии начну оформлять свои гипотезы}

\dialog{Antonio}{это было бы здорово}

\dialog{Вова}{До завтра}

\dialog{Antonio}{сладких снов, сладенький (немного пшёнистого стиля)}

\dialog{Вова}{я ж говорил, что розовый}
\end{flushleft}

{\small\texttt{Занавес.}}

{\tiny\texttt{Небольшое послесловие для щепетильных читателей: розовый = значит лесбиянистый, т.е. мне нравятся девушки. А раз я парень и мне нравятся девушки, то всё норм!\\
Не ваш Антонио)}}
