\section*{Пьеса}
\begin{center}
    { \Large Описание алгоритма написания книги }
    % Треш, угар и содомия

    Трагикомедия в трех актах.\\
\end{center}

\begin{epigraph}
    --- В нас начал пропадать дух олдфагости\ldots\\
    --- Мы перестали писать в окна любимым женщинам\ldots
    \flushright{\normalfont Вырвано из контекста разговора об ICQ, 14~октября~2016}
\end{epigraph}

Основана на реальных событиях с минимальным вкраплением художественного вымысла.\\

\begin{figure}[ht!]
    \centering
    \includegraphics[width=\textwidth]{reservoirdogs}
    \caption{Славным ублюдкам посвящается\ldots}
\end{figure}

\begin{center}
    \Large Действие первое и единственное % ибо уникальное
\end{center}

\begin{center}
    \large Пролог
\end{center}

\vspace{1ex}

\note{Четверг. День, плавно перетекающий в вечер. Октябрь. 13~число.\linebreak2016~год.}

\note{Под осенним солнцем четверо джентельменов и дама обсуждают план мероприятий на пятничный вечер. Краткая выдержка из диалога.}


\vspace{1ex}

\dialog{Anthony}{Предлагаю игру: слова--поезда. Принцип прост: последняя буква слова является первой буквой следующего. Примеры:
\begin{enumerate}
  \item нУ УдачИ И ИсполнениЯ ЯрыХ Хотений!
  \item Просто описать тебе ерунду, указав вариант теперешнего опуса.
\end{enumerate}}

\dialog{Лёха}{Если ихтиандр розового окраса --- атакуйте его!}

\dialog{Вова \remark{обращаясь к пришедшей Виктории, которую позвал Лёха}}{Оо, какие люди!}

\dialog{Вова}{Алекс, ты уверен, что у Вики достаточно устойчивая психика для этого круговорота безумия?}

\note[space]{Вика зловеще хихикает.}

\dialog{Вова}{Хотя нет, мы же плоды её воображения, и она сама это подстроила.}

\dialog{Вика \remark{тщательно подбирая слова для игры Антона}}{Какое ехидное единоборство!}

\dialog{Лёха}{Вальдемар, скоро узнаем\ldots}

\vspace{1ex}


\begin{center}
    \large Акт 1
\end{center}

\vspace{2ex}

\note{Пятница. Вечер. Октябрь. 14 число. 2016 год.}

\note{Морось. Пожалуй это лучшее слово для описания погоды, которая шаталась по улице в ту ночь, постукивая горошинами капель в окна мирно дремавших горожан. Четыре представительных джентельмена, сидя за круглым деревянным столом, режутся в преферанс и перекидываются мыслишками.}

\note{Действующие лица:\\
\begin{itemize}
  \item[] Лёха \vspace*{-2.5ex}\begin{flushright}\begin{minipage}{.7\textwidth}душа компании --- святой отец, доктор, программист и лаборант-исследователь\end{minipage}\end{flushright}
  \item[] Anthony \vspace*{-2.5ex}\begin{flushright}\begin{minipage}{.7\textwidth} мозг компании --- таксист, еврей, инженер-исследователь в прошлом и учебный мастер в настоящем\end{minipage}\end{flushright}
  \item[] Илья \vspace*{-2.5ex}\begin{flushright}\begin{minipage}{.7\textwidth} руки компании --- веб-разработчик, корректор, дизайнер и верстальщик в одном флаконе\end{minipage}\end{flushright}
  \item[] Вова \vspace*{-2.5ex}\begin{flushright}\begin{minipage}{.7\textwidth} ноги компании --- самый быстрый парень на районе, а ещё МНС и КМС по литрболу в завязке\end{minipage}\end{flushright}
\end{itemize}
}


\vspace{1ex}

\dialog{Лёха}{Anthony, я давеча нашёл прелюбопытнейшую картинку для нашей книги. \href{http://cs8.pikabu.ru/post_img/2016/10/14/5/1476431396146890414.jpg}{Не желаете взглянуть?}}

\dialog{Anthony}{Пытаюсь как-нибудь привязать шутку про Билли Милигана к этой картинке: <<И тут я осмотрел все свои личности>>. Что-то в этом духе --- надо додумать.}

\dialog{Лёха}{Что-то типа: <<Я сижу один, а меня окружают одни дураки>>?}

\dialog{Anthony}{Можно и сторону одиночной камеры заключения обыграть.}

\dialog{Лёха \remark{немного повременив}}{Написание хорошего чёрного юмора --- сложная задача.}

\dialog{Anthony}{Вообще качественной более-менее приличной шутки.}

\dialog{Лёха}{Ну в чёрном нужно ещё суметь не оскорбить всяких идиотов.}

\dialog{Лёха \remark{загадочно улыбаясь}}{Или наоборот\ldots}

\dialog{Илья \remark{смеясь}}{Тут скорее наоборот.}

\dialog{Anthony}{Точнее оскорбить так, чтобы они этого не поняли.}

\dialog{Илья}{<<Множественные дураки Билли Миллигана>>. Документальный роман Дэниела Киза.}

\dialog{Лёха}{<<Я и мои дураки>>. Автобиография Билли Милигана.}

\dialog{Илья}{Дартаньян и три дурака.}

\dialog{Лёха}{Хорошо, что дураки, а не \ldots}

\note[space]{Илья смеётся.}

\dialog{Anthony \remark{комментирует сказанное}}{Оправдательная речь директора автоваза: <<Хорошо, что дураки, а не \ldots>>.}

\dialog{Лёха}{А не брак --- здесь должна быть отсылка к нашему предыдущему разговору.} % здесь и далее в комментариях приводится оригинал фразы. (здесь должна быть отсылка к переписке в вк) % вставить ссылку на соответствующую главу в книге

\dialog{Anthony}{Предлагаешь остальным её пересказать?} % Предлагаешь её сюда завернуть?

\dialog{Лёха}{Да.}

\note[space]{Anthony достаёт из рукава жестом волшебника какую-то мятую бумажку сомнительной свежести, разворачивает так, чтобы все видели текст, и начинает читать.}

\begin{miamalist}
  \item[Лёха:] У меня такой вопрос: почему слово брак означает свадьбу и негодную деталь?
  \item[Anthony:] Может его стоит отнести к разряду нецензурных? Так как только мат люди используют и для радости, и для описания печали. % )
  \item[Лёха:] Придётся тогда писать бр*к. И все сразу --- что это за слово???!
  \item[Anthony:] Бра*, б*ак.
  \item[Лёха:] Б**к.
  \item[Anthony:] Быык. % )
  \item[Лёха:] Беек.
  \item[Anthony:] Блик.
  \item[Лёха:] Блок.
  \item[Anthony:] Баюк (котёнок Баюна).
  \item[Лёха:] И самое неочевидное --- б-звезда-звезда-к.
  \item[Anthony:] Одним словом: хорошую вещь браком не назовут. % )
  \item[Anthony:] Так говорят они. В смысле, пословицы. Хотя и люди тоже подойдут\ldots Двусмысленно: хотя и люди тоже подойдут.
  \item[Лёха:] Брак всему голова. Семь раз отмерь, один раз брак.
  \item[Anthony:] Сделал брак, иди переделывай.
  \item[Anthony:] Не откладывай на завтра то, что можешь сделать браком сегодня.
  \item[Лёха:] Брак на брак не приходится.
  \item[Anthony:] Всемогущ бог, да хитёр брак.
  \item[Лёха:] Сколько человека не корми, всё равно на брак смотрит.
  \item[Anthony:] Работа не брак, сама себя не сделает.
  \item[Лёха:] Брак браком вышибают.
  \item[Anthony:] Без труда не сделаешь брака ни черта.
  \item[Лёха:] Нет так страшен брак, как его малюют.
  \item[Anthony:] Новая подподрубрика --- бракованные идеи?
  \item[Лёха:] Поговорки на новый лад?
  \item[Anthony:] На новый брак.
  \item[Лёха:] Точно!
  \item[Anthony:] Сейчас смотрю на нашу переписку и окончательно сформулировал и без того витавшую в воздухе гипотезу: всё начинается от простейшего и эволюционирует к тому, что мы в конце будем называть идеалом --- нет такого, что вдруг --- бац --- и нате! Готово совершенство. Закон Вселенной, шах и мат.
  \item[Лёха:] Шах, мат и нате.
  \item[Anthony:] НАТЕ --- для двупрочтения. % )
  \item[Лёха:] Здесь в любом смысле хорошо звучит.
\end{miamalist}

\note[space]{В воздухе повисла минутная тишина.}

\dialog{Илья \remark{пытаясь продолжить беседу}}{
\begin{itemize}
  \item[] Не говори брак, пока не поломаешь.
  \item[] Не всё коту качество, будет и брак.
  \item[] Брак с возу --- конвейеру легче.
  \item[] В ногах брака нет.
  \item[] Глядит в книгу, видит брак.
\end{itemize}}

\dialog{Anthony}{Брак с ленты --- конвейеру легче --- это что-то вроде осовременивания пословиц.}

\dialog{Anthony}{В ногах \textbf{Болта} брака нет.}

\dialog{Лёха \remark{вставая из-за стола}}{Они же на новый лад--брак.}

\dialog{Anthony \remark{улыбаясь и отслеживая передвижения Лёхи}}{Ну да, ну да.}

\dialog{Лёха \remark{взяв с полки книгу}}{Спонсор следующей фразы --- \href{http://www.mista.ru/pogovorki.htm}{<<Сборник пословиц и поговорок>>}:
\begin{itemize}
  \item[] Без брака бракованные.
  \item[] Брак в помощь.
  \item[] Брак создал, брак и забрал.
  \item[] Брак всё стерпит.
  \item[] Была у собаки хата, брак пришел --- она сгорела.
  \item[] В ногах брака нет.
  \item[] Там хорошо, где брака нет.
  \item[] Вот где брак зарыт.
  \item[] Вывести брак на чистую воду.
  \item[] Где браки зимуют.
  \item[] Брак --- не тётка.
  \item[] Два сапога --- пара, а три --- брак.
  \item[] Дело пахнет браком.
  \item[] Брак познаётся в беде.
  \item[] Брака не хватает.
  \item[] И швец, и жнец, и бракоделец.
\end{itemize}}

\dialog{Anthony \remark{сдавая карты}}{А указывать источники как спонсоров --- интересная задумка.}

\note[space]{Лёха кладет книгу, возвращается на место.}

\dialog{Anthony \remark{после секундных раздумий продолжает}}{Надо будет её в книжку привить\ldots}

\dialog{Лёха}{Добрый доктор Антон сделает прививку и вашей книге.}

\dialog{Илья}{Книжный грипп?}

\dialog{Лёха \remark{\href{http://i5.imageban.ru/out/2014/09/04/442aff271469c9b3f514584819fcc35c.jpg}{Изображает лицом доктора Хауса}}}{Возможно, а то может быть и книжчанка.}

\dialog{Илья}{Dr. Book-us. Хм, <<Во все книжные>>. Теория большой книги.}

\dialog{Лёха}{11 литературных друзей.}

\dialog{Илья \remark{шёпотом}}{Книжки, кофе\ldots}

\dialog{Илья \remark{громче, показывая на стол}}{2 стола?}

\dialog{Лёха \remark{задумываясь}}{Шкафа, бобра, кота\ldots}

\dialog{Илья \remark{показывает большой палец вверх}}{Книжки, кофе, 2 кота!}

\note[space]{Вова возвращается в комнату с кружкой свежеразлитого бренди.}

\dialog{Вова}{Брак без водки --- деньги на ветер!}

\dialog{Вова \remark{выпивая залпом весь напиток, продолжает, слегка морщась}}{Семь раз отмерь, один раз брак. Брак браку брак уже было?}

\dialog{Anthony}{Так можно всё что угодно переделать:\\
Было у отца три сына.\\
Старший был умён,\\
средний силён,\\
а ещё один --- бракован.}

\dialog{Вова}{Было у отца три сына:\\
Старший умный был детина,\\
Средний был и так, и сяк,\\
Младший --- откровенный брак!}

\dialog{Вова \remark{блаженно улыбаясь}}{Книжки, кофе, 2 кота! --- это просто милота.}

\note[space]{Некоторое время ничего не слышно, кроме тихого шелеста карт.}

\dialog{Anthony}{Да уж, пятница определённо вышла плодотворной на идеи --- по-прежнему жду ваших иллюстраций с завтрашней философии.}

\dialog{Лёха \remark{уклончиво}}{Всё зависит от музы\ldots}

\dialog{Anthony}{От скучности лекции.}

\dialog{Вова \remark{хитро прищурившись}}{Антон, ты опять во времени запутался --- завтра ждать надо, а не по-прежнему.}

\dialog{Вова}{Future Simple вместо Present Continiuos надо.}

\dialog{Anthony}{Хочется ответить цитатой современного мёртвого--живого поэта--рэпера (Шрёдингера?) --- сегодня завтра станет вчера.}

\dialog{Вова}{С каких пор мэр Киева --- рэпер?}

\dialog{Anthony}{Это гуф --- забыл Лену?}

\dialog{Вова \remark{улыбается}}{Я такие вещи не слушаю.}

\dialog{Илья}{Кличко --- Лена?}

\dialog{Лёха \remark{отстранённо}}{Мёртвая вещь.}

\dialog{Anthony}{Очень нерекомендую, особенно перед завтраком.}

\dialog{Лёха}{Чтобы завтра не стало сегодня! Ну или не только завтра.}

\dialog{Вова \remark{разговаривая разными голосами}}{-- Не слушайте перед завтраком русский рэп.\\
-- Так, помилуйте, другого-то низкосортного говна и нет!\\
-- Вот никакое и не слушайте!}

\note[space]{Снова повисла тишина. Илья уходит на кухню за порцией бренди, Anthony сидит со скучающим видом.}

%\dialog{Anthony}{Мне кажется или стикеры --- это какая то нездоровая тема?}
\dialog{Anthony \remark{в тоске смотря в окно}}{Мне кажется или пантомима --- это какая то нездоровая тема?}

\note[space]{Лёха пожимает плечами, Вова изображает Фрейда.}

\dialog{Лёха}{Ну не на столько же!}

%\dialog{Anthony}{Вывод вечера --- беседа переходит в угар, когда в ход идут стикеры.}
\dialog{Anthony}{Вывод вечера --- беседа переходит в угар, когда в ход идёт пантомима.}

%\dialog{Лёха}{Вывод: не нужно нюхать стикеры!}
\dialog{Лёха}{Вывод: не нужно нюхать краску для мимов!}

%\dialog{Anthony}{У нас была беседа в телеграмме. Несколько идей и стикеры, но мы боялись их трогать\ldots}
\dialog{Anthony}{У нас было выступление на аллее, несколько номеров и пантомим, но мы боялись их изображать\ldots}

%\dialog{Лёха}{И куча различных смайлов всех цветов и расцветок}
\dialog{Лёха}{И куча различных свистелок всех цветов и расцветок.}

%\dialog{Anthony}{А ещё кто-то постоянно пересылал сообщения из вк}
\dialog{Anthony}{А ещё кто-то постоянно пересказывал события за прошедший день. Самокритичный ублюдок.} % )

\dialog{Вова}{У меня алиби.}

\dialog{Лёха}{А я в домике.}

\dialog{Вова}{<<Самокритичный ублюдок)>>.}

\dialog{Лёха}{Юзай картинки.} % нужна аналогия к устному разговору

\dialog{Вова}{Долго, дорого, нахуz не нужно.}

\dialog{Вова}{Предлагаю разместить эту цитату в коментариях в коде нашего форума.}

\note[space]{Вика врывается в комнату и всплёскивает руками.}

\dialog{Вика}{Вова матом ругается!}

\dialog{Лёха \remark{делая вид, что не заметил Вику, говорит тоном меланхоличного флегматика в пустоту}}{Может быть ещё ASCII артов и на каждой странице?}

\dialog{Вова \remark{улыбаясь}}{Вова так разговаривал каждым летом, когда во дворе бегал.}

\dialog{Вова \remark{подмигивая Вике}}{Это красный, детка!}

\dialog{Anthony \remark{тоном диванного эксперта}}{Вова не ругается, а ясно формулирует свои эмоции в словестных выражениях определённой направленности.}

\dialog{Anthony \remark{полушёпотом добавляет}}{Направленность снова двусмысленна.}

\dialog{Вика \remark{тоном учителя младших классов}}{Определенной нравственности.}

\note[space]{Вика уходит из комнаты с лицом, полным обреченности, по дороге чуть не сбивает Илью, вернувшегося с кружкой. Немного бренди расплёскивается на пол.}

\dialog{Anthony}{К чёрту нравственность --- только водоворот безумия --- только bookcore!}

\vspace{1ex}


\begin{center}
    \large Акт 2
\end{center}

\vspace{2ex}

\note{Пятница. Вечер сменяется ночью. 14 число. Октябрь. 2016~год.}

\note{Дождь за окном прекратился, оставив за собой на улице мелкие лужицы. Из подъездов и пивнушек стали вылазить бродяги, пьянчужки и ночные бабочки.}

\note{В комнате с круглым деревянным столом для преферанса царит явное оживление.}

\note{К действующим лицам добавляется Кот.}


\vspace{1ex}

\dialog{Лёха \remark{оживлённо}}{Анархия?}

\dialog{Вова}{Мать порядка?}

\dialog{Anthony}{А разве у нас не она?}

\dialog{Вова \remark{с ехидной улыбкой}}{Напиши, как за тобой приедут.}

\dialog{Anthony \remark{театрально закатывая глаза}}{Чёрный воронок уже вылетел!}

\dialog{Лёха}{Я могу выехать.}

\dialog{Anthony \remark{продолжает язвить}}{Сушите сухарики --- пишите мелким почерком.}

\dialog{Илья \remark{стоя у окна и поглаживая Кота}}{Переписка голубями?}

\dialog{Илья \remark{резко обернувшись, смотрит на Лёху}}{И Голубевыми?}

\dialog{Вова \remark{также смотря на Лёху}}{Алекс, меня подбери по пути.}

\dialog{Вова \remark{кивая в сторону Ильи с кружкой}}{И коньячок тоже.}

\dialog{Anthony}{Такси Лёха --- <<Я могу выехать>>.}

\dialog{Вова \remark{смеясь, встаёт из-за стола}}{<<А могу не выехать>>. Троллинг Шрёдингера.}

\dialog{Илья}{<<А могу такси вам вызвать!>>}

\dialog{Вова \remark{приплясывая гопак}}{<<А могу ментов, диктуйте адрес!>>}

\dialog{Илья \remark{напевая}}{Мой адрес сегодня такой\ldots}

\dialog{Лёха \remark{не попадая в такт}}{Пр-кт Ленина, 28, Волгоград.}

\dialog{Anthony \remark{бросает отрешённый взгляд в пространство}}{\href{http://pine-forum.herokuapp.com/}{Не дом и не улица.}}

\dialog{Вова}{Кафедра философии и права.}

\dialog{Илья}{Вызывать ментов на лекцию?}

\dialog{Лёха}{Чтобы не было скучно!}

\dialog{Anthony \remark{всё так же отрешенно}}{Пр-кт Университетский, 100 --- Кафедра английского языка --- на следующей остановке загляните.}

\dialog{Anthony \remark{собрался с мыслями}}{Через неделю.}

\note[space]{Илья изображает гангстера.}

\dialog{Илья}{Всем лежать, никому не двигаться!}

\note[space]{Вова изображает ваххабита.}

\dialog{Илья \remark{вспоминая фразу Anthony про английский язык}}{Эээ, донт мув, донт мув.}

\note[space]{Лёха изображает Че, который лихо заливается дьявольским смехом, пробирающим до самых костей жалкую плоть смертных людишек. Anthony смеётся под свой не в меру длинный и горбатый нос.}

%\dialog{Anthony}{Снова стикеры --- я сваливаю!}
\dialog{Anthony}{Снова пантомима --- я сваливаю!}

\dialog{Илья \remark{с упрёком}}{Чё ты как не мужик-то?}

\note[space]{Илья тычет в Anthony Котом.}

\dialog{Вова}{Нормально ж начинали.}

\dialog{Лёха \remark{пародирует голос Anthony}}{I don't want to live on this planet anymore.}

\dialog{Anthony}{Быть мужиком: 200.000 лет назад --- убить мамонта. 21 век --- выдержать атаку мимов.} % стикер-атаку

\dialog{Илья \remark{с арабским акцентом}}{Мим--террор!} % Стикер-бомбинг!

\dialog{Anthony \remark{задумчиво}}{Некстати говоря.}

\dialog{Лёха \remark{немного язвительно}}{Мимы подгорают?} % Стикеры подгорают?

\dialog{Anthony \remark{не замечая этого}}{Может составить этакий список мужицкости по временам или столетиям?}

\dialog{Anthony}{19 век --- быть подстреленным на дуэли и выжить!}

\dialog{Anthony \remark{машет рукой и выкидывает пятюню}}{Привет, Галуа!}

\dialog{Anthony \remark{бубнит}}{Минутка черного юмора.}

\dialog{Илья}{<<Пушкин, чё ты дохнешь, ты чё не мужик, а?>>}

\dialog{Лёха}{Привет от Галуа \remark{\href{http://i031.radikal.ru/1404/5d/e6a261c33899.gif}{Изображает крутящегося в гробу Галуа.}}}

\dialog{Anthony \remark{с радостью во вновь загоревшихся глазах}}{Наркомания от Лёхи --- всё норм --- я остаюсь!!!}

%\dialog{Вова \remark{тоном морпеха}}{Задавим их стикерами!}
\dialog{Вова \remark{тоном морпеха}}{Задавим их!}

\dialog{Лёха \remark{тоном нашкодившего Карлсона}}{Мне просто завезли свежего\ldots}

\note[space]{Все смотрят на Лёху в ожидании конца фразы.}

\dialog{Илья \remark{не стерпев}}{Медведя?}

\dialog{Вова \remark{подхватывая}}{Галуа?}

\dialog{Лёха}{Подойдёт любой вариант.}

\dialog{Лёха \remark{пародируя голос из рации}}{Радирую важную информацию: мы ушли с маршрута.}

\dialog{Anthony}{Пилот, где и куда мы маршрутировали?}

\dialog{Лёха \remark{тоном алкоголика после глубокого запоя}}{Мы на дне.}

\dialog{Илья}{Route 60?}

\dialog{Илья \remark{поспешно}}{На острове.}

\dialog{Лёха}{Где Джек?}

\dialog{Илья}{За водой пошел.}

\dialog{Илья}{Мне больше интересно: где Харли?}

\note[space]{Вова уходит из комнаты, Кот бежит за ним.}

\dialog{Илья \remark{вдогонку Вове}}{И торчок.}

\dialog{Anthony \remark{выпав из коматозной задумчивости}}{Что за? О чём вы --- я потерялся! Человек за бортом!}

\dialog{Илья \remark{смеясь}}{Тут сотни людей за бортом.}

\dialog{Лёха}{Кидай ему пакет с коксом!}

\dialog{Илья \remark{делает замах рукой}}{Пусть цифры пишет.}

\dialog{Anthony \remark{шмыгая странно носом}}{Так может корабля-то и нет, а капитан-то голый.}

\dialog{Илья \remark{ничуть не смутившись}}{До костей!}

\dialog{Anthony \remark{оглядев комнату}}{А Вова уже выехал!}

\dialog{Anthony}{Упрямый ублюдок!}

\dialog{Илья}{На философию?}

\dialog{Anthony}{Ваши прилагательные на бочку.}

\dialog{Anthony}{На философскую бочку.}

\dialog{Илья \remark{одновременно с Anthony}}{Сделай бочку!}

\dialog{Илья \remark{смутившись}}{Философское --- уже прилагательное.}

\dialog{Илья \remark{подняв глаза, настойчиво}}{Так что сделай философскую бочку!}

\dialog{Anthony \remark{пытается показать бочку на рисунке}}{\href{https://thumbs.dreamstime.com/thumb_850/8505013.jpg}{Рисунок.}}

\dialog{Лёха \remark{мельком взглянув}}{А где место для человека?}

\dialog{Anthony \remark{поясняет рисунок}}{Содержит этанол --- всё что нужно для философии.}

\dialog{Anthony \remark{в ответ на вопрос Лёхи}}{Внутри!}

\dialog{Лёха \remark{поднимает густые брови}}{А дверь тогда где?}

\dialog{Лёха}{Только не говори, что внутри.}

\dialog{Anthony \remark{голосом наркомана}}{Это загадочная философская бочка.}

\dialog{Лёха \remark{поднимает руки к небу и выдаёт}}{\href{http://sad.co.ua/wp-content/uploads/2014/07/dveri-bochka.png}{Спасибо, интернет!}}

\dialog{Anthony \remark{убито и весело одновременно}}{Дверей нет --- границ тоже --- всё дзен.}

\dialog{Anthony \remark{глядя на часы, которые показывают 22:23}}{22:22.}

\dialog{Anthony}{Блин.}

\dialog{Anthony \remark{опять бубнит}}{Через сутки надо повторить.}

\dialog{Anthony \remark{оживленно}}{Или --- вперёд на запад!}

\dialog{Лёха \remark{изображая походку ковбоя}}{На дикий запад!}

\note[space]{Пару минут спустя в комнату забегает Кот, прыгает к Лёхе на колени. В дверном проеме появляется Вова, вернувшийся из ванной.}

\dialog{Вова \remark{радостно}}{Торчок вернулся!}

\dialog{Лёха \remark{глядя на Вальдемара}}{Закинулся и вернулся?}

%\dialog{Вова}{Ага, я снова вижу стикеры}
\dialog{Вова}{Ага, я снова вижу Галуа.}

\dialog{Anthony}{Хороший мет.}

\dialog{Вова}{Я не упрямый, я больной.}

\dialog{Вова \remark{гундося}}{Чёртов насморк.}

\vspace{1ex}


\begin{center}
    \large Акт 3
\end{center}

\vspace{2ex}

\note{Пятница. Ночь. Октябрь. 14 число. 2016 год.}

\note{За окном тускло горит уличный фонарь. С деревьев падают листья, покрывая одеялом теперь уже спящих на скамейках пьянчуг. Небо, ранее заволоченное тучами, медленно проясняется, открывая взору практически полную Луну.}

\note{За столом осталось трое игроков, занятых партией в дурака. Четвертый джентельмен в комнате мирно подрёмывает на стуле с Котом на коленях.}

\note{Действующие лица: те же личности.}


\vspace{1ex}

\dialog{Вова \remark{просыпаясь}}{Люди, вы где? Чего затихли?}

\note[space]{Кот в панике убегает от Вовы к Лёхе на колени.}

\dialog{Лёха}{Просто нечего сказать по этому поводу.}

\dialog{Вова \remark{потирая глаза}}{Закончился юмор в юморницах?}

\dialog{Anthony \remark{раздосадованно}}{Кокс выветривается --- батареи не греют --- мы трезвеем!}

\dialog{Илья}{Бракованный юмор.}

\dialog{Вова}{А у меня греют.}

\dialog{Anthony \remark{с лёгкой завистью}}{Везучий ублюдок!}

\dialog{Лёха \remark{с не совсем лёгкой завистью}}{Присоединяюсь к выше сказанному.}

\dialog{Илья \remark{пытаясь посмотреть Anthony в карты}}{У нас тоже греют.}

\dialog{Anthony \remark{не обращая внимания на Илью}}{В принципе, этот термин превращает любую фразу в тарантиновскую.}

\dialog{Вова \remark{протяжно завывая}}{Ворошиловский район, ветер северный\ldots}

\dialog{Anthony \remark{подхватывает}}{Дует из окна --- зла немеренно!}

\dialog{Лёха \remark{подпевает}}{Окна затвори и зло угомони\ldots}

\dialog{Anthony \remark{поёт хип-хоп}}{Щели замажь --- печку-баньку истопи.}

\dialog{Вова \remark{присоединяется к Anthony}}{Пирог испеки, соседей накорми.}

\dialog{Лёха \remark{меняя мотив на народный}}{Собрались хозяюшки!}

\dialog{Вова}{Студень замути, немного накати.}

\dialog{Anthony \remark{c улыбкой}}{Отчаянные домохозяины.}

\dialog{Вова \remark{критично}}{Не ну так себе затея для сериала.}

\dialog{Лёха \remark{задумчиво}}{Смотрю на какую аудиторию.}

\dialog{Anthony \remark{истошно пародируя истеричку}}{Я выращиваю мандарины в Волгограде, Карл, a ты говоришь так себе идея?}

\dialog{Лёха}{Мандариновый магнат Anthony.}

\dialog{Anthony}{Антонио --- немного средиземноморья.}

\dialog{Вова}{Ты ещё скажи, что мандарин не твой, тебе подкинули.}

\dialog{Anthony \remark{изображая невинную овечку}}{Так и было --- он сам пришёл.}

\dialog{Лёха}{Вкусный, спелый, но не мой.}

\dialog{Вова}{Немой в одно слово выглядит лучше.}

\dialog{Anthony}{Немой в любом виде выглядит немного загадочно.}

\dialog{Лёха}{Это дело вкуса.}

\note[space]{Вова снова изображает Фрейда.}

\dialog{Лёха \remark{с лёгкой иронией}}{Спасибо доктор, но нам не нужна консультация.}

\dialog{Anthony \remark{прикрывает ладонью лицо, мурлыкая себе под орлиный носоклюв}}{Опять рецидив.}

\dialog{Илья \remark{возмущенно}}{Опять, вы серьезно?}

% \dialog{Вова}{*стикер с корейцем*}
\note[space]{Вова изображает корейца.}

\dialog{Anthony}{Где старые добрые олдскульные словестные аськи?}

\dialog{Илья \remark{пародируя звук сообщения}}{О-оу!}

\dialog{Илья}{Ржунимагу.}

\dialog{Вова}{*ROFL*}

\note[space]{Илья складывает руки в странную фигуру, надевая их на голову как шлем пилота.}

\dialog{Anthony \remark{ободрённо}}{Вот оно!}

\note[space]{Лёха внезапно начинает тыкать Котом в Илью.\\
Илья забирает у него Кота и начинает им тыкать в Вальдемара.\\
Лёха отбирает Кота и зачем-то тычет им в монитор выключенного компьютера.}

\dialog{Вова \remark{со столетней тоской в голосе и обречённым взглядом приговорённого к расстрелу}}{Такой олдскул намечался, а вы опять всё засрали котиками!}

\note[space]{Кот обиженно запрыгивает на подоконник.}

\dialog{Илья}{*FACEPALM*}

\dialog{Вова}{\href{https://icq.com/windows/ru}{Скатилась асечка\ldots}}

\dialog{Лёха \remark{бодрым тоном могильщика}}{Закапывайте!}

\dialog{Anthony \remark{изображая Ипполита}}{В нас начал пропадать дух олдфагости\ldots}

\dialog{Илья}{Так мыло же.}

\dialog{Вова \remark{намыливая шапку}}{Мы перестали писать в окна любимым женщинам\ldots}

\dialog{Anthony}{Фраза вечера! Недели!}

\note[space]{Илья корчится от раздирающего его смеха.}

\dialog{Лёха}{Двусмысленная.}

\dialog{Илья}{Все зависит от ударения.}

\dialog{Anthony}{Всё зависит от места удара --- ударения.}

\dialog{Илья \remark{задумчиво}}{Чак норрис ставит ударение на один и тот же слог? Мисье француз.}

\dialog{Вова}{Франсуа.}

\note[space]{У Вовы звонит телефон, остальные пытаются по-быстрому закончить партию в дурака.}

\dialog{Вова \remark{разговаривая по телефону}}{Мой брат смотрит на ютубе ролики с поняшками в озвучках на различных языках. А как проходит вечер у вас?}

\dialog{Лёха}{Я тут с какими-то людьми в преферанс зависаю.}

\dialog{Илья \remark{пытаясь понять во что же они играют}}{Слушаем, как твой брат смотрит на ютубе ролики с поняшками в озвучках на различных языках.}

\dialog{Anthony}{Господа присяжные заседатели, просто прохожие и ежи с ними --- для добивания страниц и просто для памяти --- предлагаю запилить этот вечер в книжку --- под рубрикой пятничный угар.}

\note[space]{Мысль мелькнула в голове Anthony.}

\dialog{Anthony}{Название надо для рубрики перепридумать.}

\dialog{Илья}{Пятница, четырнадцатое.}

\dialog{Вова \remark{возвращаясь за стол}}{Пятничное моё.}

\dialog{Anthony}{Ещё.}

\dialog{Лёха}{Пятницкое.}

\dialog{Илья \remark{с контонским акцентом}}{5низза?}

\dialog{Вова}{5низзя? Пять ниндзя?}

\dialog{Лёха}{Пятничные посиделки, полежанки, пописанки\ldots}

\dialog{Лёха}{Насчёт пяти ниндзя --- формально участвуют только четверо.}

\dialog{Илья}{Пописанки.}

\dialog{Вова}{Написанки.}

\dialog{Лёха}{Насиделки и належанки.}

\dialog{Вова \remark{отмахивается от летучих мышей}}{Страх и ненависть в пятницу.}

\dialog{Вова}{Назови это послесловием. Или предисловием.}

\dialog{Anthony}{Вместословием.}

\dialog{Вова}{Или приложение А: описание алгоритма написания книги.}

\dialog{Anthony \remark{Вове}}{Это интересно!}

\dialog{Вова \remark{выпятив грудь}}{Вова --- это не только 56 килограмм диетического мяса, но и всегда интересно.}

\note[space]{Партия в дурака окончена. Илья берет Кота с подоконника и занимает место, на котором ещё недавно спал Вова.}

\dialog{Anthony}{Но надо отметить в подназвании пятницу.}

\dialog{Вова}{У тебя семь пятниц на неделе.}

\dialog{Лёха \remark{смеётся}}{Хорошая наверное неделя.}

\dialog{Вова}{Алгоритм --- инвариант преобразования сдвига по времени.}

% \dialog{Вова}{Ты инициируешь чат случайным трешем из контакта и понеслась}
\dialog{Вова}{Ты инициируешь разговор случайным трешем, и понеслась.}

\dialog{Anthony}{Не случайным, а тщательно подобранной наркоманией!}

\dialog{Anthony \remark{горделиво}}{Мы тут вам не эти.}

\note[space]{Anthony тычет пальцем куда-то вверх.}

\dialog{Вова}{Тема диссертации: изучение устойчивости динамической системы с генерацией юмора.}

\dialog{Лёха}{Эмиссия юмора в вакууме.}

\dialog{Вова \remark{смеясь}}{Терморектальная!}

\dialog{Вова}{Кстати, эмиссия юмора в вакууме может происходить только в виде пантомимы!}

\dialog{Лёха}{Кто и когда это доказал?}

\dialog{Вова \remark{гордо}}{Я только что.}

\note[space]{Лёха быстро уходит из комнаты.}

\dialog{Anthony}{Гипотеза Абрахманова за нумером 97.}

\dialog{Вова}{А потом космонавты 200 лет будут выяснять --- так это или не так.}

\dialog{Anthony \remark{как менеджер в автосалоне}}{Не имеется желания замутить 100 абдрагипотез?}

\dialog{Вова}{Абдрипотез.}

\dialog{Вова \remark{сомневаясь}}{Ну не знаю, они ж качественные должны быть\ldots}

\dialog{Вова \remark{бубнит}}{Покачественнее идей для стартапа.}

\dialog{Anthony \remark{как матёрый менеджер в автосалоне}}{Серьёзно? Тебя не ограничивают рамки. Можно не сто!}

\note[space]{Anthony изображает Фрейда.}

\dialog{Anthony}{69!}

\dialog{Вова}{Не, максимум 13.}

\dialog{Anthony}{Давай!}

\dialog{Вова}{При этом не будет 4, 9 и 13.}

\dialog{Anthony}{Зарубим отдельную главу. Или даже книгу в книге в книге\ldots}

\dialog{Вова \remark{усмехаясь}}{А первая гипотеза --- Гипотез 4, 9 и 13 не существует!}

\dialog{Anthony \remark{задумавшись}}{Но количественно их всё равно будет 13?}

\dialog{Вова}{Точнее, гипотез 4, 9 и d не существует.}

\dialog{Anthony}{Так лучше.}

\dialog{Вова}{Ну, в итоге их 10 останется.}

\note[space]{Лёха заходит в комнату с плюшкой в руке.}

\dialog{Anthony}{Лёха вернулся!}

\dialog{Лёха \remark{голосом Фрейкенбок}}{Туточки я.}

\dialog{Anthony \remark{задумчиво}}{Хотя восстал звучит пафосней.}

\dialog{Anthony \remark{к Вове}}{Цитрусовый ублюдок, ещё 9.}

\dialog{Вова}{Бесславные ублюдки.}

\dialog{Anthony}{Безцельный.}

\dialog{Лёха}{Пока бесславные.}

\dialog{Anthony}{Славные ублюдки!}

\dialog{Anthony}{Милоты немного.}

\dialog{Лёха \remark{тянется за Котом}}{Опять котиков?}

\dialog{Anthony}{Нее.}

\dialog{Вова}{Ну можно ещё как в бешеных псах --- я мистер оранжевый, Антон --- розовый, Алекс --- зелёный.}

\dialog{Anthony \remark{обиженно}}{Почему я розовый?}

\dialog{Лёха \remark{дожевывая плюшку}}{А я не смотрел.}

\dialog{Anthony \remark{всё еще обиженно}}{Был же чёрным на форуме?}

\dialog{Вова}{А фон??}

\dialog{Вова}{Кстати, Anthony, ты процитировал фразу из фильма!}

\dialog{Anthony}{Необъективно.}

\dialog{Anthony \remark{громко}}{И зачем ограничиваться видимым диапазоном? Мы, блин, физики или где?}

\dialog{Вова \remark{поёт в стиле Буратино}}{УЛЬ!}

\dialog{Вова \remark{громче}}{ТРА!!}

\dialog{Вова \remark{ещё громче}}{ФИО!!!}

\dialog{Вова \remark{кричит}}{ЛЕТ!!!!}

\note[space]{Илья просыпается, но тут же засыпает. Кот возвращается на подоконник. Лёха встаёт и начинает собираться на выход.}

\dialog{Anthony \remark{полушёпотом}}{456 нм.}

\dialog{Вова}{Ты хвастаешься?}

\dialog{Anthony \remark{смеясь}}{Bовчика сегодня можно на цитаты записывать!}

\dialog{Вова}{У меня сегодня бенефис.}

\dialog{Лёха \remark{прощаясь}}{Ладно я мухожук. И до завтра!}

\dialog{Вова}{Давай, Владимирыч, до завтра!}

\dialog{Вова \remark{разочарованно}}{А я только разогрелся\ldots}

\dialog{Anthony}{Итак, они сошлись --- вода и пламень. Лёд и камень.}

\dialog{Anthony}{Антон и Вова.}

\dialog{Вова}{Эт хреново.}

\note[space]{Вова растряс Илью, оба прошли в прихожку, обуваются.}

\dialog{Вова \remark{Anthony}}{Ладно, завтра увидимся!}

\dialog{Anthony}{Последний герой!}

\dialog{Вова \remark{задумчиво}}{Может я на философии начну оформлять свои гипотезы.}

\dialog{Anthony \remark{тоном менеджера в автосалоне, только что продавшего автомобиль}}{Это было бы здорово!}

\dialog{Вова}{До завтра.}

\dialog{Anthony \remark{пародируя женский голос}}{Сладких снов, сладенький!} %(немного пшёнистого стиля)}

\dialog{Вова}{Я ж говорил, что розовый.}


\note[space]{Занавес.}

{\tiny\texttt{Небольшое послесловие для щепетильных читателей: розовый --- значит лесбиянистый, то есть мне нравятся девушки. А раз я парень и мне нравятся девушки, то всё норм!\\
Не ваш Антонио)}}
