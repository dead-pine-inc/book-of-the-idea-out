\section*{Пьеса}
\subsection*{Описание алгоритма написания книги}
\subsubsection*{Треш, угар и содомия} % зачёркнуто
\begin{epigraph}
    --- В нас начал пропадать дух олдфагости...\\
    --- Мы перестали писать в окна любимым женщинам...
    \flushright{\normalfont Вырвано из контекста разговора об ICQ, 14 октября 2016}
\end{epigraph}

Основана на реальных событиях с минимальным вкраплением художественного вымысла.\\ 

\begin{figure}[ht!]
    \centering
    \includegraphics[width=\textwidth]{reservoirdogs}
    \caption{Славным ублюдкам посвящается...}
\end{figure}

\begin{center}
    \Large Действие первое и единственное % ибо уникальное
\end{center}

{\small\texttt{Пятница. Вечер. Октябрь. 14 число. 2016 год.}}

{\small\texttt{Морось. Пожалуй это лучшее слово для описания погоды, которая шаталась по улице в ту ночь, постукивая горошинами капель в окна мирно дремавших горожан. 4 представительных джентельмена, сидя за круглым деревянным столом, режутся в преферанс и перекидываются мыслишками.}}

\begin{center}
    \large Part 1
\end{center}

\begin{flushleft}
\textbf{[Anthony]}: Предлагаю игру: слова-поезда.\\
Принцип прост: последняя буква слова = первой следующего\\
Примеры:\\
1) нУ УдачИ И ИсполнениЯ ЯрыХ Хотений!\\
2) Просто описать тебе ерунду, указав вариант теперешнего опуса.

\textbf{[Лёха]}: Если ихтиандр розового окраса -- атакуйте его!

\textbf{[Vova, обращаясь к вошедшей в комнату Виктории, которую позвал Лёха]}: оО, какие люди!

\textbf{[Vova]}: Алекс, ты уверен, что у Вика достаточно устойчивая психика для этого круговорота безумия?

\textbf{[Вика зловеще хихикает]}

\textbf{[Vova]}: Хотя нет, мы же плоды её воображения, и она сама это подстроила.

[Вика (видимо о чём то своём нам неведомом)]: Какое ехидное единоборство!

\textbf{[Лёха]}: Вольдемар, скоро узнаем...

\textbf{[Лёха]}: Антуан, я давеча нашёл прелюбопытнейшую картинку для нашей книги. \href{http://cs8.pikabu.ru/post_img/2016/10/14/5/1476431396146890414.jpg}{Не желаете взглянуть?}

\textbf{[Anthony]}: Пытаюсь как-нибудь привязать шутку про Билли Милигана к этой картинки: "И тут я осмотрел все свои личности" (что то в этом духе -- надо додумать)

\textbf{[Лёха]}: Что-то типа: "Я сижу один, а меня окружают одни дураки"?

\textbf{[Anthony]}: Можно и сторону одиночной камеры заключения обыграть.

\textbf{[Лёха]}: Написание хорошего чёрного юмора -- сложная задача.

\textbf{[Anthony]}: Вообще качественной более-менее приличной шутки.

\textbf{[Лёха]}: Ну в чёрном нужно ещё суметь не оскорбить всяких идиотов.

\textbf{[Лёха, загадочно улыбаясь]}: Или наоборот...

\textbf{[Илья, улыбаясь ещё более загадочно]}: Тут скорее наоборот.

\textbf{[Anthony]}: Точнее оскорбить так, чтобы они этого не поняли.

\textbf{[Илья]}: "Множественные дураки Билли Миллигана"?
"Документальный роман Дэниела Киза"

\textbf{[Лёха]}: "Я и мои дураки". Автобиография Билли Милигана

\textbf{[Илья]}: Дартаньян и три дурака

\textbf{[Лёха]}: Хорошо, что дураки, а не ...

\textbf{[Илья смеётся]}

\textbf{[Anthony комментирует вышесказанное]}: Оправдательная речь директора автоваза

"Хорошо, что дураки, а не ..."

\textbf{[Лёха]}: А не брак (здесь должна быть отсылка к переписке в вк) % вставить ссылку на соответствующую главу в книге

\textbf{[Anthony]}: Предлагаешь её сюда завернуть?

\textbf{[Лёха]}: Да.

\textbf{[Anthony достаёт из рукава жестом волшебника какую то мятую бумажку сомнительной свежести и начинает читать]}: У меня такой вопрос: почему слово брак означает свадьбу и негодную деталь?

\begingroup
\fontfamily{ccr}\selectfont
Anton 10:06 pm % ту надо переоформить как то нормально. ТеХ редактор, мне нужна твоя помощь!

Может его стоит отнести к разряду нецензурных? Т.к. только мат люди используют и для радости , и для описания печали)

\textbf{[Alexey, 10:07 pm]}: Придётся тогда писать бр*к

И все сразу -- что это за слово???!

\textbf{[Anton, 10:07 pm]}: бра*

б*ак

\textbf{[Alexey, 10:07 pm]}: б**к

\textbf{[Anton, 10:08 pm]}: быык)

\textbf{[Alexey, 10:08 pm]}: беек

\textbf{[Anton, 10:08 pm]}: блик

\textbf{[Alexey, 10:08 pm]}: блок

\textbf{[Anton, 10:09 pm]}: баюк

(котёнок Баюна)

\textbf{[Alexey, 10:09 pm]}: и самое неочевидное

б-звезда-звезда-к

\textbf{[Anton, 10:10 pm]}: одним словом: хорошую вещб браком не назовут)

так говорят они

в смысле, пословицы

хотя и люди тоже подойдут

2-х смысленно

\textbf{[Forwarded Messages, Anton 13.10.16]}: хотя и люди тоже подойдут

\textbf{[Alexey, 10:12 pm]}: брак всему голова

семь раз отмерь, один раз брак

\textbf{[Anton, 10:12 pm]}: сделал брак, иди переделывай

\textbf{[Anton, 10:13 pm]}: не откладывай на завтра то, что можешь сделать браком сегодня

\textbf{[Alexey, 10:13 pm]}: брак на брак не приходится

\textbf{[Anton, 10:14 pm]}: Всемогущ бог, да хитёр брак

\textbf{[Alexey, 10:15 pm]}: Сколько человека не корми, всё равно на брак смотрит

\textbf{[Anton, 10:15 pm]}: работа не брак, сама себя не сделает

\textbf{[Alexey, 10:16 pm]}: брак браком вышибают

\textbf{[Anton, 10:17 pm]}: без труда не сделаешь брака ни черта

\textbf{[Alexey, 10:18 pm]}: нет так страшен брак, как его малюют

\textbf{[Anton, 10:19 pm]}: новая подподрубрика?

бракованные идеи

\textbf{[Alexey, 10:19 pm]}: поговорки на новый лад?

\textbf{[Anton, 10:19 pm]}: на новый брак

\textbf{[Alexey, 10:19 pm]}: точно!

\textbf{[Anton, 10:22 pm]}: сейчас смотрю на нашу переписку и окончательно сформулировал и без того витавшую в воздухе гипотезу

всё начинается от простейшего и эволюционирует к тому, что мы в конце будем называть идеалом -- нет такого, что вдруг -- бац -- и нате! готово совершенство

закон Вселенной

шах и мат

\textbf{[Alexey, 10:23 pm]}: шах, мат и нате

\textbf{[Anton, 10:24 pm]}: НАТЕ -- для двупрочтения)

\textbf{[Alexey, 10:24 pm]}: здесь в любом смысле хорошо звучит
\endgroup

\textbf{[Илья]}: не говори брак, пока не поломаешь

не всё коту качество, будет и брак

брак с возу -- конвейеру легче

в ногах брака нет

глядит в книгу, видит брак

\textbf{[Anthony]}: брак с ленты -- конвейеру легче -- что-то вроде осовременивания пословиц

\textbf{[Anthony]}: в ногах \textbf{[Болта]} брака нет

\textbf{[Лёха]}: Они же на новый лад/брак.

\textbf{[Anthony, улыбаясь]}: Ну да. Ну да.

\textbf{[Лёха]}: Спонсор следующего текста http://www.mista.ru/pogovorki.htm

Без брака бракованные.

Брак в помощь.

Брак создал, брак и забрал.

Брак всё стерпит.

Была у собаки хата, брак пришел -- она сгорела.

В ногах брака нет.

Там хорошо, где брака нет.

Вот где брак зарыт.

Вывести брак на чистую воду.

Где браки зимуют.

Брак -- не тётка.

Два сапога -- пара, а три -- брак.

Дело пахнет браком.

Брак познаётся в беде.

Брака не хватает.

И швец, и жнец и бракоделец.

\textbf{[Anthony]}: А указывать ссылки как спонсоров --- интересная задумка.

\textbf{[Anthony, после секундных раздумий, продолжает]}: Надо будет её в книжку привить...

\textbf{[Лёха]}: Добрый доктор Антон сделает прививку и вашей книге.

\textbf{[Илья]}: Книжный грипп?

[Лёха \href{http://i5.imageban.ru/out/2014/09/04/442aff271469c9b3f514584819fcc35c.jpg}{изображает лицом доктора Хауса}]: Возможно, а может быть и книжчанка.

\textbf{[Илья]}: dr. Book-us

\textbf{[Илья]}: Хм, "Во все книжные"

\textbf{[Илья]}: Теория большой книги.

\textbf{[Лёха]}: 11 литературных друзей.

\textbf{[Илья шёпотом]}: книжки, кофе

\textbf{[Илья громче]}: 2 стола?

\textbf{[Лёха]}: шкафа, бобра, кота...

\textbf{[Илья показывает большой палец вверх]}: Книжки, кофе, 2 кота!

\textbf{[Vova возвращаясь в комнату с кружкой свежеразлитого бренди]}: Брак без водки -- деньги на ветер!

\textbf{[Vova, выпивая залпом весь напиток, продолжает, слегка морщась]}: Семь раз отмерь, один раз брак.\\

Брак браку брак уже было?

\textbf{[Anthony]}: Так можно всё что угодно переделать: Было у отца 3 сына. Старший был умён, средний силён, а ещё один -- бракован.

\textbf{[Vova]}: Было у отца три сына:\\
Старший умный был детина,\\
Средний был и так, и сяк,\\
Младший -- откровенный брак!

\textbf{[Vova, блаженно улыбаясь]}: Это просто милота.

\textbf{[Anthony]}: Да уж пятница определённо вышла плодотворной на идеи --- попрежнему жду ваших иллюстраций с завтрашней философии.

[Лёха (уклончиво)]: Всё зависит от музы...

\textbf{[Anthony]}: От скучности лекции.

\textbf{[Vova, хитро прищурившись]}: Антон, ты опять во времени запутался -- завтра ждать надо, а не по-прежнему.

\textbf{[Vova]}: Future Simple вместо Present Continiuos надо.

\textbf{[Anthony]}: Хочется ответить цитатой современного мёртвого/живого поэта/рэпера (шрёдингера?) --- сегодня завтра станет вчера.

\textbf{[Vova]}: С каких пор мэр Киева -- репер?

\textbf{[Anthony]}: Это гуф -- забыл Лену?

\textbf{[Vova ехидно улыбается]}: Я такие вещи не слушаю.

\textbf{[Илья]}: Кличко = Лена?

[Лёха (не совсем ясно о ком/чём?)]: Мёртвая вещь.

\textbf{[Anthony]}: Очень нерекомендую) особенно перед завтраком.

\textbf{[Лёха]}: Чтобы завтра не стало сегодня!\\
Ну или не только завтра.

\textbf{[Vova]}: -- Не слушайте перед завтраком русский рэп.\\
— Так, помилуйте, другого-то низкосортного говна и нет!\\
— Вот никакое и не слушайте!

\textbf{[Anthony]}: Мне кажется или стикеры --- это какая то нездоровая тема?

\textbf{[Vova изображает Фрейда]}: 

\textbf{[Лёха]}: Ну не на столько же!

\textbf{[Anthony]}: Вывод вечера --- беседа переходит в угар, когда в ход идут стикеры.

\textbf{[Лёха]}: Вывод: не нужно нюхать стикеры!

\textbf{[Anthony]}: у нас была беседа в телеграмме. несколько идей и стикеры, но мы боялись их трогать...

\textbf{[Лёха]}: и куча различных смайлов всех цветов и расцветок

\textbf{[Anthony]}: а ещё кто-то постоянно пересылал сообщения из вк

\textbf{[Anthony]}: самокритичный ублюдок)

\textbf{[Vova]}: У меня алиби.

\textbf{[Лёха]}: А я в домике.

\textbf{[Vova]}: "самокритичный ублюдок)"

\textbf{[Лёха]}: юзай картинки

\textbf{[Vova]}: долго, дорого, нахуz не нужно.

\textbf{[Vova]}: Предлагаю разместить эту цитату в коментариях в коде нашего форума.

\textbf{[Вика врывается в комнату и всплёскивает руками]}: Вова матом ругается!

\textbf{[Лёха тоном меланхоличного флегматика]}: Может быть ещё ASCII артов и в каждой странице?

\textbf{[Vova]}: Вова так разговаривал каждым летом, когда во дворе бегал)

[Очередной ох-вдох от Вики]: 

\textbf{[Vova, подмигивая]}: Это красный, детка!

\textbf{[Anthony тоном диванного эксперта]}: Вова не ругается, а ясно формулирует свои эмоции в словестных выражениях определённой направленности;

\textbf{[Anthony полушёпотом добавляет]}: направленность снова 2смысленна.

\textbf{[ТОном учителя младших классов Виктория]}: Определенной нравственности*

\textbf{[Anthony]}: К чёрту нравственность --- Только водоворот безумия --- только bookcore!
\end{flushleft}

\begin{center}
    \large Part 2
\end{center}

{\small\texttt{Всё те же лица + кот.}}

\begin{flushleft}
\textbf{[Лёха оживлённо]}: Анархия?

\textbf{[Vova]}: Мать порядка?

\textbf{[Anthony]}: А разве у нас не она?

\textbf{[Vova с ехидной улыбкой]}: Напиши, как за тобой приедут.

\textbf{[Anthony, театрально закатывая глаза]}: Чёрный воронок уже вылетел!

\textbf{[Лёха]}: Я могу выехать.

\textbf{[Anthony продолжает язвить]}: Сушите сухарики --- пишите мелким почерком.

\textbf{[Илья]}: Переписка голубями?

\textbf{[Илья добавляет]}: и Голубевыми?

\textbf{[Vova]}: Алекс, меня подбери по пути.

\textbf{[Vova]}: и коньячок тоже)

\textbf{[Anthony]}: Такси Лёха --- "Я могу выехать"

\textbf{[Vova]}: "А могу не выехать".

\textbf{[Forwarded from Abdra Vova]}: Троллинг Шрёдингера.

\textbf{[Илья]}: "А могу такси вам вызвать!"

\textbf{[Vova, приплясывая гопак]}: "А могу ментов, диктуйте адрес!"

\textbf{[Илья]}: Мой адрес сегодня такой:

\textbf{[Лёха]}: Пр-кт Ленина, 28, Волгоград,

\textbf{[Anthony бросает отрешённый взгляд в пространство]}: \href{http://pine-forum.herokuapp.com/}{не дом и не улица}

\textbf{[Vova]}: кафедра философии и права

\textbf{[Илья]}: ментов на лекцию?

\textbf{[Лёха]}: Чтобы не было скучно!

\textbf{[Anthony]}: Пр-кт Университетский, 100 --- Кафедра английского языка --- на следующей остановке загляните;

\textbf{[Anthony]}: через неделю.

\textbf{[Илья изображает гангстера]}: Всем лежать, никому не двигаться!

\textbf{[Vova изображает ваххабита]}

\textbf{[Илья напевает]}: эээ, донт мув, донт мув

\textbf{[Лёха изображает Че, который лихо заливается дьявольским смехом, пробирающим до самых костей жалкую плоть смертных людишек]}

\textbf{[Anthony, смеясь под свой не в меру длинный и горбатый нос]}: Снова стикеры --- я сваливаю!

\textbf{[Илья с упрёком]}: Чё ты как не мужик-то?

\textbf{[Илья тычет в Anthony котом]}

\textbf{[Vova]}: Нормально ж начинали.

\textbf{[Лёха пародирует голос Anthony]}: *I don't live this planet anymore*

\textbf{[Anthony]}: Быть мужиком: 200.000 лет назад --- убить мамонта. 21 век --- выдержать стикер-атаку.

\textbf{[Илья поправляет]}: стикер-бомбинг!

\textbf{[Задумчиво Anthony]}: Некстати говоря

\textbf{[Лёха немного язвительно]}: Стикеры подгорают?

\textbf{[Anthony не замечая этого]}: может составить этакий список мужикости по временам/столетиям?

\textbf{[Anthony]}: 19 век -- быть подстреленным на дуэли и выжить!

\textbf{[Anthony машет рукой и выкидывает пятюню]}: Привет, Галуа!

\textbf{[Anthony бубнит]}: минутка черного юмора

\textbf{[Илья]}: "Пушкин, чё ты дохнешь, чё не мужик?"

\textbf{[Лёха]}: Привет от Галуа

\textbf{[изображает крутящегося в гробу Галуа]}

\textbf{[Anthony с радостью во вновь загоревшихся глазах]}: Наркомания от Лёхи --- всё норм --- я остаюсь)

\textbf{[Vova угрожающи]}: Задавим их стикерами!

\textbf{[Тоном нашкодившего Карлсона Лёха]}: Мне просто завезли свежего ...

\textbf{[Илья]}: Медведя?

\textbf{[Vova]}: Галуа?

\textbf{[Лёха]}: Подойдёт любой вариант

\textbf{[Forwarded from Лёха]}: радирую важную информацию: мы ушли с маршрута

\textbf{[Anthony]}: Пилот, где/куда мы маршрутировали?

\textbf{[Лёха тоном алкоголика после запоя недели в 2]}: мы на дне

\textbf{[Илья]}: route 60?

\textbf{[Илья]}: на острове

\textbf{[Лёха]}: Где Джек?

\textbf{[Илья]}: за водой пошел

\textbf{[Илья]}: мне больше интересно

\textbf{[Илья]}: где Харли?

\textbf{[Илья]}: и торчок

\textbf{[Выпав из коматозной задумчивости Anthony]}: что за ? о чём вы --- я потерялся) Человек за бортом!

\textbf{[Смеясь Илья]}: Тут сотни людей за бортом.

\textbf{[Лёха]}: Кидай ему пакет с коксом!

\textbf{[Илья делает замах рукой]}: Пусть цифры пишет.

\textbf{[Шмыгая странно носом Anthony]}: так может корабля то и нет, а капитан то голый)

\textbf{[Ничють не смутившись Илья]}: до костей!

\textbf{[Anthony оглядел комнату]}: А Вова уже выехал!

\textbf{[Посмеиваясь Anthony]}: Упрямый ублюдок!

\textbf{[Илья]}: На философию?

\textbf{[Anthony]}: ваши прилагательные на бочку

\textbf{[Anthony]}: на философскую бочку

\textbf{[Илья]}: сделай бочку!

\textbf{[Илья]}: философское -- уже прилагательное

\textbf{[Илья]}: т.ч. сделай философскую бочку!

\textbf{[Anthony пытается показать бочку на рисунке]}: \href{https://thumbs.dreamstime.com/thumb_850/8505013.jpg}{Рисунок}

\textbf{[Менльком взглянув Лёха]}: А где место для человека?

\textbf{[Anthony поясняет]}: содержит этанол --- всё что нужно для философии

\textbf{[Anthony]}: внутри

\textbf{[Лёха поднимает густые брови]}: а дверь тогда где?

\textbf{[Лёха]}: только не говори, что внутри

\textbf{[Голосом наркомана Anthony]}: это загадочная философская бочка

\textbf{[Лёха поднимает руки к небу и выдаёт]}: \href{http://sad.co.ua/wp-content/uploads/2014/07/dveri-bochka.png}{спасибо интернет}

\textbf{[Anthony убито и весело одновременно]}: дверей нет --- границ тоже --- всё дзен

\textbf{[Anthony, глядя на часы, которые показывают 22:23]}: 22:22

\textbf{[Anthony]}: блин

\textbf{[Anthony опять бубнит]}: через сутки надо повторить

\textbf{[Оживая через минуту Anthony]}: или --- вперёд на запад!

\textbf{[Vova выскользнул из ванной]}: Торчок вернулся!

\textbf{[Лёха изображает походку ковбоя]}: На дикий запад!

\textbf{[Лёха, глядя на Вольдемара]}: @citrux закинулся и вернулся?

\textbf{[Vova]}: ага, я снова вижу стикеры

\textbf{[Anthony]}: хороший мет

\textbf{[Vova]}: я не упрямый, я больной

\textbf{[Vova]}: чёртов насморк
\end{flushleft}

\begin{center}
    \large Part 3
\end{center}

{\small\texttt{Пятница. Ночь. Октябрь. 15 число. 2016 год.}}

{\small\texttt{Те же личности.}}

\begin{flushleft}
\textbf{[Vova, который до этого слегка закимарил]}: Люди, вы где?

\textbf{[Vova]}: Чего затихли?

\textbf{[Лёха]}: просто нечего сказать по этому поводу

\textbf{[Vova]}: Закончился юмор в юморницах?

\textbf{[Anthony]}: кокс выветривается -- батареи не греют -- мы трезвеем)

\textbf{[Илья]}: бракованный юмор

\textbf{[Vova]}: А у меня греют

\textbf{[С лёгкой завистью Anthony]}: везучий ублюдок)

\textbf{[Лёха с не совсем лёгкой завистью]}: присоединяюсь к выше сказанному

\textbf{[Илья]}: у нас тоже греют

\textbf{[Anthony]}: в прнципе этот термин превращает любую фразу в тарантиновскую

\textbf{[Vova протяжно завывает]}: Ворошиловский район, ветер северный

\textbf{[Anthony подхватывает]}: дует из окна -- зла немерено)

\textbf{[Лёха]}: Окна затвори и зло угомони

\textbf{[Anthony]}: щели замажь -- печку / баньку истопи

\textbf{[Vova]}: Пирог испеки, соседей накорми

\textbf{[Лёха]}: Собрались хозяюшки!

\textbf{[Vova]}: Студень замути, немного накати

\textbf{[Anthony c улыбкой]}: отчаянные домохозяины

\textbf{[Vova, критично]}: Не ну так себе затея для сериала

\textbf{[Лёха, задумчиво]}: Смотрю на какую аудиторию

\textbf{[Anthony, истошно пародируя истеричку]}: Я выращимаю мандарины в Волгограде, Карл, a ты говоришь так себе идея?

\textbf{[Лёха]}: Мандариновый магнат Антон

\textbf{[Anthony]}: Антонио --- немного средиземноморья

\textbf{[Vova]}: Ты ещё скажи, что мандарин не твой, тебе подкинули

\textbf{[Антонио]}: так и было --- он сам пришёл

\textbf{[Лёха]}: Вкусный, спелый, но не мой.

\textbf{[Vova]}: немой в одно слово выглядит лучше

\textbf{[Антонио]}: немой в любом виде выглядит немного загадочно

\textbf{[Лёха]}: это дело вкуса

\textbf{[Vova снова изображает Фрейда]}

\textbf{[Лёха с лёгкой иронией]}: Спасибо доктор, но нам не нужна консультация.

\textbf{[Антонио прикрывает ладонью лицо, мурлыкая себе под орлиный носоклюв]}: опять рецидив

\textbf{[Илья]}: Опять, вы серьезно?

\textbf{[Vova]}: *стикер с корейцем*

\textbf{[Антонио]}: Где старые добрые олдскульные словестные аськи?

\textbf{[Илья]}: О-оу

\textbf{[Илья]}: Ржунимагу

\textbf{[Vova]}: *ROFL*

\textbf{[Илья]}: B)

\textbf{[Антонио ободрённо]}: вот оно!

\textbf{[Лёха начинает тыкать котом в Илью]}

\textbf{[Илья забирает у него кота и начинает им тыкать в Вольдемара]}

\textbf{[Лёха отбирает кота и зачем то тычет им в монитор выключенного компьютера]}

\textbf{[Vova cj столетней тоской в голосе и обречённым взглядом приговорённого к расстрелу]}: такой олдскул намечался, а вы опять всё засрали котиками

\textbf{[Илья]}: *FACEPALM*

\textbf{[Vova]}: \href{https://icq.com/windows/ru}{скатилась асечка...}

\textbf{[Бодрым тоном могильщика Лёха]}: Закапывайте

\textbf{[Изображая Ипполита Antonio]}: в нас начал пропадать дух олдфагости...

\textbf{[Илья]}: так мейл же

\textbf{[Vova]}: мы перестали писать в окна любимым женщинам....

\textbf{[Antonio]}: фраза вечера)\\
недели!

\textbf{[Илья корчится от заздирающего его смеха]}

\textbf{[Лёха]}: 2х смысленная

\textbf{[Илья]}: Все зависит от ударения

\textbf{[Antonio]}: Всё зависит от места удара (ударения)

\textbf{[Илья]}: Чак норрис ставит ударение на один и тот же слог?

\textbf{[Илья]}: Мисье француз

\textbf{[Vova]}: франсуа

\textbf{[Vova]}: Мой брат смотрит на ютубе ролики с поняшками в озвучках на различных языках. А как проходит вечер у вас?

\textbf{[Лёха]}: Я тут с какими-то людьми в преферанс зависаю.

\textbf{[Илья]}: Слушаем, как твой брат смотрит на ютубе ролики с поняшками в озвучках на различных языках

\textbf{[Antonio]}: Господа, присяжные заседатели, просто прохожие и ежи с ними --- для добивания страниц и просто для памяти --- предлагаю запилить этот вечер в книжку --- под рубрикой пятничный угар

\textbf{[Чуть подумав Antonio добавил]}: название надо для рубрики перепридумать

\textbf{[Илья]}: Пятница, 14е

\textbf{[Vova]}: Пятничное моё

\textbf{[Antonio]}: ещё

\textbf{[Лёха]}: Пятницкое

\textbf{[Илья с контонским акцентом]}: 5низза?

\textbf{[Vova]}: 5низзя?\\
5 ниндзя?\\

\textbf{[Лёха]}: Пятничные посиделки|полежанки|пописанки|...

\textbf{[Лёха]}: насчёт 5 ниндзя -- формально участвуют только 4

\textbf{[Илья]}: Пописанки

\textbf{[Vova]}: написанки

\textbf{[Лёха]}: насиделки и належанки

\textbf{[Vova отмахивается от летучих мышей]}: Страх и ненависть в пятницу

\textbf{[Vova]}: Назови это послесловием
или предисловием

\textbf{[Antonio]}: вместословием

\textbf{[Vova]}: или приложение А: описание алгоритма написания книги

\textbf{[Antonio]}: @citrux это интересно

\textbf{[Vova]}: @citrux -- это не только 56 килограмм диетического мяса, но и всегда интересно

\textbf{[Antonio]}: но надо отметить в подназвании пятницу

\textbf{[Vova]}: у тебя 7 пятниц на неделе

\textbf{[Лёха]}: хорошая наверное неделя

\textbf{[Vova]}: алгоритм -- инвариант преобразования сдвига по времени

\textbf{[Vova]}: ты инициируешь чат случайным трешем из контакта и понеслась

\textbf{[Antonio]}: не случайным, а тщательно подобранной наркоманией

\textbf{[Antonio]}: мы тут вам не эти

\textbf{[Vova]}: Тема диссертации: изучение устойчивости динамической системы с генерацией юмора

\textbf{[Лёха]}: Эмиссия юмора в вакууме

\textbf{[Vova]}: терморектальная)

\textbf{[Vova]}: кстати, эмиссия юмора в вакууме может происходить только в виде пантомимы)

\textbf{[Лёха]}: Кто и когда это доказал?

\textbf{[Vova]}: Я только что

\textbf{[Antonio]}: Гипотеза Абрахманова за нумером 97

\textbf{[Vova]}: а потом космонавты 200 лет будут выяснять -- так это или не так

\textbf{[Antonio]}: не имеется желания замутить 100 абдрагипотез?

\textbf{[Vova]}: абдрипотез

\textbf{[Vova]}: ну не знаю, они ж качественные должны быть

\textbf{[Vova]}: покачественнее идей для стартапа

\textbf{[Antonio]}: серьёзно?\\
тебя не граничивают рамки\\
можно не сто\\

\textbf{[Antonio изображает Фрейда]}: 69

\textbf{[Vova]}: не, максимум 13

\textbf{[Antonio]}: давай

\textbf{[Vova]}: при этом не будет 4, 9 и 13

\textbf{[Antonio]}: зарубим отдельную главу
или даже книгу в книге в книге...

\textbf{[Vova]}: А первая гипотеза -- Гипотез 4, 9 и 13 не существует

\textbf{[Antonio]}: но количественно их всё = будет 13?

\textbf{[Vova]}: точнее, гипотез 4, 9  и d не существует

\textbf{[Antonio]}: так лучше

\textbf{[Vova]}: ну в итоге их 10 останется

\textbf{[Antonio]}: @FreeCX вернулся

\textbf{[Лёха голосом Фрейкенбок]}: туточки я

\textbf{[Antonio задумчиво]}: хотя восстал звучит пафосней

\textbf{[Antonio к Вове]}: цитрусовый ублюдок, ещё 9

\textbf{[Vova]}: бесславные ублюдки

\textbf{[Antonio]}: безцельный

\textbf{[Лёха]}: пока бесславные)

\textbf{[Antonio]}: славные ублюдки

\textbf{[Antonio]}: милоты немного

\textbf{[Лёха]}: опять котиков?

\textbf{[Antonio]}: нее

\textbf{[Vova]}: ну можно ещё как в бешеных псах -- я мистер оранжевый, Антон -- розовый, Алекс -- зелёный

\textbf{[Antonio]}: почему я розовый?

\textbf{[Лёха]}: а я не смотрел

\textbf{[Antonio]}: был же чёрным на форуме?

\textbf{[Vova]}: А фон?)

\textbf{[Vova]}: кстати, Тони, ты процитировал фразу из фильма

\textbf{[Antonio]}: необъективно

\textbf{[Antonio]}: и зачем ограничиваться видимым диапазоном? мы, блин, физики или где?

\textbf{[Vova поёт в стиле Буратино]}: УЛЬ

\textbf{[Vova]}: ТРА

\textbf{[Vova]}: ФИО

\textbf{[Vova]}: ЛЕТ!!!

\textbf{[Antonio]}: 456 нм

\textbf{[Vova]}: Ты хвастаешься?

\textbf{[Antonio]}: Bовчика сегодня можно на цитаты записывать)

\textbf{[Vova]}: У меня сегодня бенефис)

\textbf{[Лёха]}: ладно я мухожук\\
и до завтра

\textbf{[Vova]}: Давай, Владимирыч, до завтра

\textbf{[Vova]}: а я только разогрелся...

\textbf{[Antonio]}: итак они сошлись -- вода и пламень. лёд и камень)

\textbf{[Antonio]}: анто и вова

\textbf{[Vova]}: эт хреново

\textbf{[Vova]}: Ладно, завтра увидимся

\textbf{[Antonio]}: Последний герой!

\textbf{[Vova]}: Может я на философии начну оформлять свои гипотезы

\textbf{[Antonio]}: это было бы здорово

\textbf{[Vova]}: До завтра

\textbf{[Antonio]}: сладких снов, сладенький (немного пшёнистого стиля)

\textbf{[Vova]}: я ж говорил, что розовый
\end{flushleft}

{\small\texttt{Занавес.}}

{\tiny\texttt{Небольшое послесловие для щепетильных читателей: розовый = значит лесбиянистый, т.е. мне нравятся девушки. А раз я парень и мне нравятся девушки, то всё норм!\\
Не ваш Антонио)}}
