\section*{Пьеса}
\subsection*{Описание алгоритма написания книги}
\subsubsection*{\sout{Треш, угар и содомия}} % зачёркнуто
\begin{epigraph}
    --- В нас начал пропадать дух олдфагости...\\
    --- Мы перестали писать в окна любимым женщинам...
    \flushright{\normalfont Вырвано из контекста разговора об ICQ, 14 октября 2016}
\end{epigraph}

Основана на реальных событиях с минимальным вкраплением художественного вымысла.\\ 

\begin{figure}[ht!]
    \centering
    \includegraphics[width=\textwidth]{reservoirdogs}
    \caption{Славным ублюдкам посвящается...}
\end{figure}

\begin{center}
\Large Действие первое и единственное % ибо уникальное
\end{center}


{\small\textttt{Пятница. Вечер. Октябрь. 14 число. 2016 год.}}


{\small\textttt{Морось. Пожалуй это лучшее слово для описания погоды, которая шаталась по улице в ту ночь, постукивая горошинами капель в окна мирно дремавших горожан. 4 представительных джентельмена, сидя за круглым деревянным столом, режутся в преферанс и перекидываются мыслишками.}}

\begin{center}
\large Part 1
\end{center}


[Anthony]\\
Предлагаю игру: слова-поезда.\\
Принцип прост: последняя буква слова = первой следующего\\
Примеры:\\
1) нУ УдачИ И ИсполнениЯ ЯрыХ Хотений!\\
2) Просто описать тебе ерунду, указав вариант теперешнего опуса.


[Лёха]\\
Если ихтиандр розового окраса — атакуйте его!


[Vova, обращаясь к вошедшей в комнату Виктории, которую позвал Лёха]\\
оО, какие люди!


[Vova]\\
Алекс, ты уверен, что у Вика достаточно устойчивая психика для этого круговорота безумия?


[Вика зловеще хихикает]


[Vova]\\
Хотя нет, мы же плоды её воображения, и она сама это подстроила.


[Вика (видимо о чём то своём нам неведомом)]\\
Какое ехидное единоборство!


[Лёха]\\
Вольдемар, скоро узнаем...


[Лёха]\\
Антуан, я давеча нашёл прелюбопытнейшую картинку для нашей книги. \href{http://cs8.pikabu.ru/post_img/2016/10/14/5/1476431396146890414.jpg}{Не желаете взглянуть?}


[Anthony]\\
Пытаюсь как-нибудь привязать шутку про Билли Милигана к этой картинки: "И тут я осмотрел все свои личности" (что то в этом духе - надо додумать)


[Лёха]\\
Что-то типа: "Я сижу один, а меня окружают одни дураки"?


[Anthony]\\
Можно и сторону одиночной камеры заключения обыграть.


[Лёха]\\
Написание хорошего чёрного юмора — сложная задача.


[Anthony]\\
Вообще качественной более-менее приличной шутки.


[Лёха]\\
Ну в чёрном нужно ещё суметь не оскорбить всяких идиотов.


[Лёха, загадочно улыбаясь]\\
Или наоборот...


[Илья, улыбаясь ещё более загадочно]\\
Тут скорее наоборот.


[Anthony]\\
Точнее оскорбить так, чтобы они этого не поняли.


[Илья]\\
"Множественные дураки Билли Миллигана"?
"Документальный роман Дэниела Киза"


[Лёха]\\
"Я и мои дураки". Автобиография Билли Милигана


[Илья]\\
Дартаньян и три дурака


[Лёха]\\
Хорошо, что дураки, а не ...


[Илья смеётся]\\


[Anthony комментирует вышесказанное]\\
Оправдательная речь директора автоваза
"Хорошо, что дураки, а не ..."


[Лёха]\\
А не брак (здесь должна быть отсылка к переписке в вк) % вставить ссылку на соответствующую главу в книге


[Anthony]\\
Предлагаешь её сюда завернуть?


[Лёха]\\
Да.


[Anthony достаёт из рукава жестом волшебника какую то мятую бумажку сомнительной свежести и начинает читать]\\
У меня такой вопрос: почему слово брак означает свадьбу и негодную деталь?

Anton 10:06 pm % ту надо переоформить как то нормально. ТеХ редактор, мне нужна твоя помощь!

    Может его стоит отнести к разряду нецензурных? Т.к. только мат люди используют и для радости , и для описания печали)

Alexey 10:07 pm

    Придётся тогда писать бр*к
    И все сразу — что это за слово???!

Anton 10:07 pm

    бра*
    б*ак

Alexey 10:07 pm

    б**к

Anton 10:08 pm

    быык)

Alexey 10:08 pm

    беек

Anton 10:08 pm

    блик

Alexey 10:08 pm

    блок

Anton 10:09 pm

    баюк
    (котёнок Баюна)

Alexey 10:09 pm

    и самое неочевидное

    б-звезда-звезда-к

Anton 10:10 pm

    одним словом: хорошую вещб браком не назовут)
    так говорят они
    в смысле, пословицы
    хотя и люди тоже подойдут
    2-х смысленно
    Forwarded Messages
    Anton 13.10.16
        хотя и люди тоже подойдут

Alexey 10:12 pm

    брак всему голова
    семь раз отмерь, один раз брак

Anton 10:12 pm

    сделал брак, иди переделывай

Anton 10:13 pm

    не откладывай на завтра то, что можешь сделать браком сегодня

Alexey 10:13 pm

    брак на брак не приходится

Anton 10:14 pm

    Всемогущ бог, да хитёр брак

Alexey 10:15 pm

    Сколько человека не корми, всё равно на брак смотрит

Anton 10:15 pm

    работа не брак, сама себя не сделает

Alexey 10:16 pm

    брак браком вышибают

Anton 10:17 pm

    без труда не сделаешь брака ни черта

Alexey 10:18 pm

    нет так страшен брак, как его малюют

Anton 10:19 pm

    новая подподрубрика?
    бракованные идеи

Alexey 10:19 pm

    поговорки на новый лад?

Anton 10:19 pm

    на новый брак

Alexey 10:19 pm

    точно!

Anton 10:22 pm

    сейчас смотрю на нашу переписку и окончательно сформулировал и без того витавшую в воздухе гипотезу
    всё начинается от простейшего и эволюционирует к тому, что мы в конце будем называть идеалом - нет такого, что вдруг - бац - и нате! готово совершенство
    закон Вселенной
    шах и мат

Alexey 10:23 pm

    шах, мат и нате

Anton 10:24 pm

    НАТЕ - для двупрочтения)

Alexey 10:24 pm

    здесь в любом смысле хорошо звучит


[Илья]\\
не говори брак, пока не поломаешь
не всё коту качество, будет и брак
брак с возу — конвейеру легче
в ногах брака нет
глядит в книгу, видит брак


[Anthony]\\
брак с ленты — конвейеру легче - что-то вроде осовременивания пословиц


[Anthony]\\
в ногах [Болта] брака нет


[Лёха]\\
Они же на новый лад/брак.


[Anthony, улыбаясь]\\
Ну да. Ну да.


[Лёха]\\
Спонсор следующего текста http://www.mista.ru/pogovorki.htm\\

Без брака бракованные.
Брак в помощь.
Брак создал, брак и забрал.
Брак всё стерпит.
Была у собаки хата, брак пришел — она сгорела.
В ногах брака нет.
Там хорошо, где брака нет.
Вот где брак зарыт.
Вывести брак на чистую воду.
Где браки зимуют.
Брак - не тётка.
Два сапога - пара, а три - брак.
Дело пахнет браком.
Брак познаётся в беде.
Брака не хватает.
И швец, и жнец и бракоделец.


[Anthony]\\
А указывать ссылки как спонсоров --- интересная задумка.


[Anthony, после секундных раздумий, продолжает]\\
Надо будет её в книжку привить...


[Лёха]\\
Добрый доктор Антон сделает прививку и вашей книге.


[Илья]\\
Книжный грипп?


[Лёха \href{http://i5.imageban.ru/out/2014/09/04/442aff271469c9b3f514584819fcc35c.jpg}{изображает лицом доктора Хауса}]\\
Возможно, а может быть и книжчанка.


[Илья]\\
dr. Book-us


[Илья]\\
Хм, "Во все книжные"


[Илья]\\
Теория большой книги.


[Лёха]\\
11 литературных друзей.


[Илья шёпотом]\\
книжки, кофе


[Илья громче]\\
2 стола?


[Лёха]\\
шкафа, бобра, кота...


[Илья показывает большой палец вверх]\\
Книжки, кофе, 2 кота!


[Vova возвращаясь в комнату с кружкой свежеразлитого бренди]\\
Брак без водки — деньги на ветер!


[Vova, выпивая залпом весь напиток, продолжает, слегка морщась]\\
Семь раз отмерь, один раз брак.\\
Брак браку брак уже было?


[Anthony]\\
Так можно всё что угодно переделать: Было у отца 3 сына. Старший был умён, средний силён, а ещё один - бракован.


[Vova]\\
Было у отца три сына:\\
Старший умный был детина,\\
Средний был и так, и сяк,\\
Младший — откровенный брак!


[Vova, блаженно улыбаясь]\\
Это просто милота.


[Anthony]\\
Да уж пятница определённо вышла плодотворной на идеи --- попрежнему жду ваших иллюстраций с завтрашней философии.


[Лёха (уклончиво)]\\
Всё зависит от музы...


[Anthony]\\
От скучности лекции.


[Vova, хитро прищурившись]\\
Антон, ты опять во времени запутался — завтра ждать надо, а не по-прежнему.


[Vova]\\
Future Simple вместо Present Continiuos надо.


[Anthony]\\
Хочется ответить цитатой современного мёртвого/живого поэта/рэпера (шрёдингера?) --- сегодня завтра станет вчера.


[Vova]\\
С каких пор мэр Киева — репер?


[Anthony]\\
Это гуф - забыл Лену?


[Vova ехидно улыбается]\\
Я такие вещи не слушаю.


[Илья]\\
Кличко = Лена?


[Лёха (не совсем ясно о ком/чём?)]\\
Мёртвая вещь.


[Anthony]\\
Очень нерекомендую) особенно перед завтраком.


[Лёха]\\
Чтобы завтра не стало сегодня!\\
Ну или не только завтра.


[Vova]\\
— Не слушайте перед завтраком русский рэп.\\
— Так, помилуйте, другого-то низкосортного говна и нет!\\
— Вот никакое и не слушайте!


[Anthony]\\
Мне кажется или стикеры --- это какая то нездоровая тема?


[Vova изображает Фрейда]\\


[Лёха]\\
Ну не на столько же!


[Anthony]\\
Вывод вечера --- беседа переходит в угар, когда в ход идут стикеры.


[Лёха]\\
Вывод: не нужно нюхать стикеры!


[Anthony]\\
у нас была беседа в телеграмме. несколько идей и стикеры, но мы боялись их трогать...


[Лёха]\\
и куча различных смайлов всех цветов и расцветок


[Anthony]\\
а ещё кто-то постоянно пересылал сообщения из вк


[Anthony]\\
самокритичный ублюдок)


[Vova]\\
У меня алиби.


[Лёха]\\
А я в домике.


[Vova]\\
"самокритичный ублюдок)"



[Лёха]\\
юзай картинки


[Vova]\\
долго, дорого, нахуz не нужно.


[Vova]\\
Предлагаю разместить эту цитату в коментариях в коде нашего форума.


[Вика врывается в комнату и всплёскивает руками]\\
Вова матом ругается!


[Лёха тоном меланхоличного флегматика]\\
Может быть ещё ASCII артов и в каждой странице?


[Vova]\\
Вова так разговаривал каждым летом, когда во дворе бегал)


[Очередной ох-вдох от Вики]\\


[Vova, подмигивая]\\
Это красный, детка!


[Anthony тоном диванного эксперта]\\
Вова не ругается, а ясно формулирует свои эмоции в словестных выражениях определённой направленности;


[Anthony полушёпотом добавляет]\\
направленность снова 2смысленна.


[ТОном учителя младших классов Виктория]\\
Определенной нравственности*


[Anthony]\\
К чёрту нравственность --- Только водоворот безумия --- только bookcore!


\begin{center}
\large Part 2
\end{center}


{\small\textttt{Всё те же лица + кот.}}


[Лёха оживлённо]\\
Анархия?


[Vova]\\
Мать порядка?


[Anthony]\\
А разве у нас не она?


[Vova с ехидной улыбкой]\\
Напиши, как за тобой приедут.


[Anthony, театрально закатывая глаза]\\
Чёрный воронок уже вылетел!


[Лёха]\\
Я могу выехать.


[Anthony продолжает язвить]\\
Сушите сухарики --- пишите мелким почерком.


[Илья]\\
Переписка голубями?


[Илья добавляет]\\
и Голубевыми?


[Vova]\\
Алекс, меня подбери по пути.


[Vova]\\
и коньячок тоже)


[Anthony]\\
Такси Лёха --- "Я могу выехать"


[Vova]\\
"А могу не выехать".


[Forwarded from Abdra Vova]\\
Троллинг Шрёдингера.


[Илья]\\
"А могу такси вам вызвать!"


[Vova, приплясывая гопак]\\
"А могу ментов, диктуйте адрес!"


[Илья]\\
Мой адрес сегодня такой:


[Лёха]\\
Пр-кт Ленина, 28, Волгоград,


[Anthony бросает отрешённый взгляд в пространство]\\
\href{http://pine-forum.herokuapp.com/}{не дом и не улица}


[Vova]\\
кафедра философии и права


[Илья]\\ 
ментов на лекцию?


[Лёха]\\
Чтобы не было скучно!


[Anthony]\\
Пр-кт Университетский, 100 --- Кафедра английского языка --- на следующей остановке загляните;


[Anthony]\\
через неделю.


[Илья изображает гангстера]\\
Всем лежать, никому не двигаться!


[Vova изображает ваххабита]


[Илья напевает]\\
эээ, донт мув, донт мув


[Лёха изображает Че, который лихо заливается дьявольским смехом, пробирающим до самых костей жалкую плоть смертных людишек]


[Anthony, смеясь под свой не в меру длинный и горбатый нос]\\
Снова стикеры --- я сваливаю!


[Илья с упрёком]\\
Чё ты как не мужик-то?


[Илья тычет в Anthony котом]


[Vova]\\
Нормально ж начинали.


[Лёха пародирует голос Anthony]\\
*I don't live this planet anymore*


[Anthony]\\
Быть мужиком: 200.000 лет назад --- убить мамонта. 21 век --- выдержать стикер-атаку.


[Илья поправляет]\\
стикер-бомбинг!


[Задумчиво Anthony]\\
Некстати говоря


[Лёха немного язвительно]\\
Стикеры подгорают?


[Anthony не замечая этого]\\
может составить этакий список мужикости по временам/столетиям?


[Anthony]\\
19 век - быть подстреленным на дуэли и выжить!


[Anthony машет рукой и выкидывает пятюню]\\
Привет, Галуа!


[Anthony бубнит]\\
минутка черного юмора


[Илья]\\
"Пушкин, чё ты дохнешь, чё не мужик?"


[Лёха]\\
Привет от Галуа
[изображает крутящегося в гробу Галуа]


[Anthony с радостью во вновь загоревшихся глазах]\\
Наркомания от Лёхи --- всё норм --- я остаюсь)


[Vova угрожающи]\\
Задавим их стикерами!


[Тоном нашкодившего Карлсона Лёха]\\
Мне просто завезли свежего ...


[Илья]\\
Медведя?


[Vova]\\
Галуа?


[Лёха]\\
Подойдёт любой вариант


[Forwarded from Лёха]\\
радирую важную информацию: мы ушли с маршрута


[Anthony]\\
Пилот, где/куда мы маршрутировали?


[Лёха тоном алкоголика после запоя недели в 2]\\
мы на дне


[Илья]\\
route 60?


[Илья]\\
на острове


[Лёха]\\
Где Джек?


[Илья]\\
за водой пошел


[Илья]\\
мне больше интересно


[Илья]\\
где Харли?


[Илья]\\
и торчок


[Выпав из коматозной задумчивости Anthony]\\
что за ? о чём вы --- я потерялся) Человек за бортом!


[Смеясь Илья]\\
Тут сотни людей за бортом.


[Лёха]\\
Кидай ему пакет с коксом!


[Илья делает замах рукой]\\
Пусть цифры пишет.


[Шмыгая странно носом Anthony]\\
так может корабля то и нет, а капитан то голый)


[Ничють не смутившись Илья]\\
до костей!


[Anthony оглядел комнату]\\
А Вова уже выехал!


[Посмеиваясь Anthony]\\
Упрямый ублюдок!


[Илья]\\
На философию?


[Anthony]\\
ваши прилагательные на бочку


[Anthony]\\
на философскую бочку


[Илья]\\
сделай бочку!


[Илья]\\
философское — уже прилагательное


[Илья]\\
т.ч. сделай философскую бочку!


[Anthony пытается показать бочку на рисунке]\\
[\href{https://thumbs.dreamstime.com/thumb_850/8505013.jpg}{Рисунок}]


[Менльком взглянув Лёха]\\
А где место для человека?


[Anthony поясняет]\\
содержит этанол --- всё что нужно для философии


[Anthony]\\
внутри


[Лёха поднимает густые брови]\\
а дверь тогда где?


[Лёха]\\
только не говори, что внутри


[Голосом наркомана Anthony]\\
это загадочная философская бочка


[Лёха поднимает руки к небу и выдаёт]\\
\href{http://sad.co.ua/wp-content/uploads/2014/07/dveri-bochka.png}{спасибо интернет}


[Anthony убито и весело одновременно]\\
дверей нет --- границ тоже --- всё дзен


[Anthony, глядя на часы, которые показывают 22:23]\\
22:22


[Anthony]\\
блин


[Anthony опять бубнит]\\
через сутки надо повторить


[Оживая через минуту Anthony]\\
или --- вперёд на запад!


[Vova выскользнул из ванной]\\
Торчок вернулся!


[Лёха изображает походку ковбоя]\\
На дикий запад!


[Лёха, глядя на Вольдемара]\\
@citrux закинулся и вернулся?


[Vova]\\
ага, я снова вижу стикеры


[Anthony]\\
хороший мет


[Vova]\\
я не упрямый, я больной


[Vova]\\
чёртов насморк


\begin{center}
\large Part 3
\end{center}

{\small\textttt{Пятница. Ночь. Октябрь. 15 число. 2016 год.}}


{\small\textttt{Те же личности.}}


[Vova, который до этого слегка закимарил]\\
Люди, вы где?


[Vova]\\
Чего затихли?


[Лёха]\\
просто нечего сказать по этому поводу


[Vova]\\
Закончился юмор в юморницах?


[Anthony]\\
кокс выветривается - батареи не греют - мы трезвеем)


[Илья]\\
бракованный юмор


[Vova]\\
А у меня греют


[С лёгкой завистью Anthony]\\
везучий ублюдок)


[Лёха с не совсем лёгкой завистью]\\
присоединяюсь к выше сказанному


[Илья]\\
у нас тоже греют


[Anthony]\\
в прнципе этот термин превращает любую фразу в тарантиновскую


[Vova протяжно завывает]\\
Ворошиловский район, ветер северный


[Anthony подхватывает]\\
дует из окна - зла немерено)


[Лёха]\\
Окна затвори и зло угомони


[Anthony]\\
щели замажь - печку / баньку истопи


[Vova]\\
Пирог испеки, соседей накорми


[Лёха]\\
Собрались хозяюшки!


[Vova]\\
Студень замути, немного накати


[Anthony c улыбкой]\\
отчаянные домохозяины


[Vova, критично]\\
Не ну так себе затея для сериала


[Лёха, задумчиво]\\
Смотрю на какую аудиторию


[Anthony, истошно пародируя истеричку]\\
Я выращимаю мандарины в Волгограде, Карл, a ты говоришь так себе идея?


[Лёха]\\
Мандариновый магнат Антон


[Anthony]\\
Антонио --- немного средиземноморья


[Vova]\\
Ты ещё скажи, что мандарин не твой, тебе подкинули


[Антонио]\\
так и было --- он сам пришёл


[Лёха]\\
Вкусный, спелый, но не мой.


[Vova]\\
немой в одно слово выглядит лучше


[Антонио]\\
немой в любом виде выглядит немного загадочно


[Лёха]\\
это дело вкуса


[Vova снова изображает Фрейда]


[Лёха с лёгкой иронией]\\
Спасибо доктор, но нам не нужна консультация.


[Антонио прикрывает ладонью лицо, мурлыкая себе под орлиный носоклюв]\\
опять рецидив


[Илья]\\
Опять, вы серьезно?


[Vova]\\
*стикер с корейцем*


[Антонио]\\
Где старые добрые олдскульные словестные аськи?


[Илья]\\
О-оу


[Илья]\\
Ржунимагу


[Vova]\\
*ROFL*


[Илья]\\
B)


[Антонио ободрённо]\\
вот оно!

[Лёха начинает тыкать котом в Илью]


[Илья забирает у него кота и начинает им тыкать в Вольдемара]


[Лёха отбирает кота и зачем то тычет им в монитор выключенного компьютера]


[Vova cj столетней тоской в голосе и обречённым взглядом приговорённого к расстрелу]\\
такой олдскул намечался, а вы опять всё засрали котиками


[Илья]\\
*FACEPALM*

[Vova]\\
\href{https://icq.com/windows/ru}{скатилась асечка...}


[Бодрым тоном могильщика Лёха]\\
Закапывайте


[Изображая Ипполита Antonio]\\
в нас начал пропадать дух олдфагости...


[Илья]\\
так мейл же


[Vova]\\
мы перестали писать в окна любимым женщинам....


[Antonio]\\
фраза вечера)\\
недели!


[Илья корчится от заздирающего его смеха]


[Лёха]\\
2х смысленная


[Илья]\\
Все зависит от ударения


[Antonio]\\
Всё зависит от места удара (ударения)


[Илья]\\
Чак норрис ставит ударение на один и тот же слог?


[Илья]\\
Мисье француз


[Vova]\\
франсуа


[Vova]\\
Мой брат смотрит на ютубе ролики с поняшками в озвучках на различных языках. А как проходит вечер у вас?


[Лёха]\\
Я тут с какими-то людьми в преферанс зависаю.


[Илья]\\
Слушаем, как твой брат смотрит на ютубе ролики с поняшками в озвучках на различных языках


[Antonio]\\
Господа, присяжные заседатели, просто прохожие и ежи с ними --- для добивания страниц и просто для памяти --- предлагаю запилить этот вечер в книжку --- под рубрикой пятничный угар


[Чуть подумав Antonio добавил]\\
название надо для рубрики перепридумать


[Илья]\\
Пятница, 14е


[Vova]\\
Пятничное моё


[Antonio]\\
ещё


[Лёха]\\
Пятницкое


[Илья с контонским акцентом]\\
5низза?


[Vova]\\
5низзя?\\
5 ниндзя?\\


[Лёха]\\
Пятничные посиделки|полежанки|пописанки|...


[Лёха]\\
насчёт 5 ниндзя — формально участвуют только 4


[Илья]\\
Пописанки


[Vova]\\
написанки


[Лёха]\\
насиделки и належанки


[Vova отмахивается от летучих мышей]\\
Страх и ненависть в пятницу


[Vova]\\
Назови это послесловием
или предисловием


[Antonio]\\
вместословием


[Vova]\\
или приложение А: описание алгоритма написания книги


[Antonio]\\
@citrux это интересно


[Vova]\\
@citrux — это не только 56 килограмм диетического мяса, но и всегда интересно


[Antonio]\\
но надо отметить в подназвании пятницу


[Vova]\\
у тебя 7 пятниц на неделе


[Лёха]\\
хорошая наверное неделя


[Vova]\\
алгоритм — инвариант преобразования сдвига по времени


[Vova]\\
ты инициируешь чат случайным трешем из контакта и понеслась


[Antonio]\\
не случайным, а тщательно подобранной наркоманией


[Antonio]\\
мы тут вам не эти


[Vova]\\
Тема диссертации: изучение устойчивости динамической системы с генерацией юмора


[Лёха]\\
Эмиссия юмора в вакууме


[Vova]\\
терморектальная)


[Vova]\\
кстати, эмиссия юмора в вакууме может происходить только в виде пантомимы)


[Лёха]\\
Кто и когда это доказал?


[Vova]\\
Я только что


[Antonio]\\
Гипотеза Абрахманова за нумером 97


[Vova]\\
а потом космонавты 200 лет будут выяснять — так это или не так


[Antonio]\\
не имеется желания замутить 100 абдрагипотез?


[Vova]\\
абдрипотез


[Vova]\\
ну не знаю, они ж качественные должны быть


[Vova]\\
покачественнее идей для стартапа


[Antonio]\\
серьёзно?\\
тебя не граничивают рамки\\
можно не сто\\


[Antonio изображает Фрейда]\\
69


[Vova]\\
не, максимум 13


[Antonio]\\
давай


[Vova]\\
при этом не будет 4, 9 и 13


[Antonio]\\
зарубим отдельную главу
или даже книгу в книге в книге...


[Vova]\\
А первая гипотеза — Гипотез 4, 9 и 13 не существует


[Antonio]\\
но количественно их всё = будет 13?


[Vova]\\
точнее, гипотез 4, 9  и d не существует


[Antonio]\\
так лучше


[Vova]\\
ну в итоге их 10 останется


[Antonio]\\
@FreeCX вернулся


[Лёха голосом Фрейкенбок]\\
туточки я


[Antonio задумчиво]\\
хотя восстал звучит пафосней


[Antonio к Вове]\\
цитрусовый ублюдок, ещё 9


[Vova]\\
бесславные ублюдки


[Antonio]\\
безцельный


[Лёха]\\
пока бесславные)


[Antonio]\\
славные ублюдки


[Antonio]\\
милоты немного


[Лёха]\\
опять котиков?


[Antonio]\\
нее


[Vova]\\
ну можно ещё как в бешеных псах — я мистер оранжевый, Антон — розовый, Алекс — зелёный


[Antonio]\\
почему я розовый?


[Лёха]\\
а я не смотрел


[Antonio]\\
был же чёрным на форуме?


[Vova]\\
А фон?)


[Vova]\\
кстати, Тони, ты процитировал фразу из фильма


[Antonio]\\
необъективно


[Antonio]\\
и зачем ограничиваться видимым диапазоном? мы, блин, физики или где?


[Vova поёт в стиле Буратино]\\
УЛЬ


[Vova]\\
ТРА


[Vova]\\
ФИО


[Vova]\\
ЛЕТ!!!


[Antonio]\\
456 нм


[Vova]\\
Ты хвастаешься?


[Antonio]\\
Bовчика сегодня можно на цитаты записывать)


[Vova]\\
У меня сегодня бенефис)


[Лёха]\\
ладно я мухожук\\
и до завтра


[Vova]\\
Давай, Владимирыч, до завтра


[Vova]\\
а я только разогрелся...


[Antonio]\\
итак они сошлись - вода и пламень. лёд и камень)


[Antonio]\\
анто и вова


[Vova]\\
эт хреново


[Vova]\\
Ладно, завтра увидимся


[Antonio]\\
Последний герой!


[Vova]\\
Может я на философии начну оформлять свои гипотезы


[Antonio]\\
это было бы здорово


[Vova]\\
До завтра


[Antonio]\\
сладких снов, сладенький (немного пшёнистого стиля)


[Vova]\\
я ж говорил, что розовый


{\small\textttt{Занавес.}}


{\tiny\textttt{Небольшое послесловие для щепетильных читателей: розовый = значит лесбиянистый, т.е. мне нравятся девушки. А раз я парень и мне нравятся девушки, то всё норм!\\
Не ваш Антонио)}}
