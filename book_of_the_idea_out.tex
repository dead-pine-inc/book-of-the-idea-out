\documentclass{ideas}
\usepackage[utf8]{inputenc}
\usepackage[russian]{babel}
\usepackage{color}
\usepackage{amsmath}
\usepackage{amsfonts}
\usepackage{amssymb}
\usepackage{graphicx}

\renewcommand{\author}{Dead Pine, Inc.}
\usepackage[unicode, colorlinks, linkcolor=blue, citecolor=blue,
urlcolor=blue]{hyperref}
\usepackage{csquotes} % ещё одна штука для цитат
\usepackage{rotating}

\graphicspath{{images/}}

\renewcommand*\rmdefault{cmr}
\begin{document}
\begin{titlepage}

\vspace*{\fill}
\begin{center}
{\fontfamily{antt}\selectfont\Huge МЫСЛИ}    
\end{center}
\vspace*{\fill}
\begin{center}
\author\\\the\year
\end{center}
\end{titlepage}
\section*{Вместо предисловия}
\vspace*{\fill}
    От создателей без\emph{цели}ра \href{https://antoniii.github.io/}{100 идей для стартапа}.
    Они вернулись чтобы творить... но лень!\\

    \emph{\Huge{ZZZ}\LARGE{ZZZ}\Large{ZZZ}\large{ZZZ}ZZZ\small{ZZZ}\footnotesize{ZZZ}\scriptsize{ZZZ}\tiny{ZZZ}\tiny{zzz}}

\vfill
\begin{center}
Сотрудники \author:
  \begin{itemize}
    \item Алекс ---  автор-исследователь и {\TeX}нический редактор
    \item Антонио --- автор-критик и \rotatebox[origin=c]{180}{\tiny промышленный шпион}
    \item Вальдемар --- автор-стилист и \( \pi\iota\zeta o \nu \) компании
    \item Илиа --- автор-эстет и {\fontfamily{antt}\selectfont\scshape художественный} редактор
  \end{itemize}

\vfill
Благодарности:
  \begin{itemize}
    \item Серхио за роль пассивно-заочного критика
    \item Николаю Васильевичу за его великую повесть <<Записки сумасшедшего>>
  \end{itemize}
\end{center}
\vfill
\newpage
\emph{Однажды к Эйнштейну пришёл журналист.\\
--- Куда вы записываете свои мысли? --- спросил он. --- У вас есть для этого блокнот или записная книжка?\\
Эйнштейн ответил.\\
--- Милый мой! Настоящие мысли приходят в голову так редко, что их нетрудно и запомнить.}
\begin{figure}[ht!]
    \centering
    \includegraphics[width=\textwidth]{ideas}
\end{figure}
\newpage
\section*{А что ты сделал для hip-hop'a в свои годы?}\label{section:one}
\begin{displayquote}
\begin{flushright}
    \emph{Мы обожаем книги мёртвых наркоманов}\\
    Рөстәм Баян улы Булатов, 2 июня 2015
\end{flushright}
\end{displayquote}
Зачем всё это? Попытка создать новый жанр в литературном творчестве.
\begin{figure}[ht!]
    \centering
    \includegraphics[width=\textwidth]{hip-zen}
\end{figure}
 А если без пафоса, то это пародия, попытка стёба потуг оных. Ибо ныне излишне много развелось всяких псевдописателей (не поминая уже армию разномастных блоггеров).
\section{Идеи для стартапа}
Стартап --- это хобби, приносящее заработок.

\begin{figure}[ht!]
    \centering
    \includegraphics[width=\textwidth]{magnet_alarm_bed}
    \caption{Магнитный будильник. Версия \( 1.054571800(13) \)}
\end{figure}

\begin{itemize}
\item Приложение для телефона. Составлять 2-,4-стишья на иностранном языке из имеющегося набора слов для улучшения обучения.\\
\begin{flushleft}
\begin{verse}
If you get in a pub\\
And you have a sullen face ---\\
Then don't stand up\\
From out your place!
\end{verse}

\begin{verse}
Do you walk to a park?\\
Go round a place of dark!
\end{verse}
\end{flushleft}
\item Острые безопасные ножи для общепита.
\end{itemize}


\section{Идеи для хобби}
Хобби --- это способ уйти от скуки.
\begin{itemize}
\item Собирать идеи для хобби
\item Собрать библиотеку из самых странных (по содержанию, автору, оформлению и т.д.) книг из когда-либо выпущенных человечеством.
\item Искать забавные представления занятным числам:
  \begin{itemize}
    \item  $42 = 2^5 + 2 \cdot 5$
    \item  $145 = 1! + 4! + 5!$
    \item  $1729 = 19 \cdot 91 = 1^3 + 12^3 = 9^3 + 10^3$
    \item  $22 = 16_{16}$
  \end{itemize}
\item Придумывать скороговорки, начинающиеся/заканчивающиеся на одну букву:
\begin{flushleft}
\begin{verse}
Виолончелисты ввалились в вагон,\\
Виолончели выпали вкось.\\
Виолончелисты вышли вон,\\
Виолончели валяются врозь.\\

Виолончелисты в Вирджинии вдоль\\
Виолончели в ведро водрузили.\\
Виолончелисты внимательно вдаль\\
Виолончели вновь вукатили!\\

"Вертай всё взад" - вернувшись веолончедисты все вскричат.\\
Виолончель внутри вся влажная ---\\
Восвояси вскорь выпроважена.
\end{verse}
\end{flushleft}
\end{itemize}

\section{Подумать о/об...}
\begin{itemize}
\item ... написании брошюры: "Размышление о бренности бытия в поезде/автомобиле/самолёте. Самоучитель для домохозяек."
\item ... создании фирмы: "Профилактические люли: Быстро! Эффективно! Недорого!"
\item ...  создании методики по определению реального уровня образования конкретного человека (польза для начальников при подборе персонала). Основание -- анализ ответов на простые детские вопросы.
Например:
    \begin{itemize}
        \item что такое число?
        \item почему небо синее, а облака белые?
        \item куда девается грипп летом?
        \item откуда так много пород собак?
        \item почему пицца круглого, а коробка квадратного сечения?
        \item какова молния на вкус?
        \item почему люди такие идиоты?
    \end{itemize}
\end{itemize}


\section{Рубрика n-смысленности + The game of words}
\begin{epigraph}
        --- Пойдём до комсы или до чекистов?\\
        --- Просто пошли, а там как пойдёт.\\
        Из разговора двух прохожих, 3 сентября 2016
\end{epigraph}

\begin{figure}[ht!]
    \centering
    \includegraphics[width=0.6\textwidth]{N}
    \caption{Mathumor}
\end{figure}

Классика двухсмысленности:
\begin{itemize}
    \item гонять чаи
    \item заварит кашу
    \item бросаться в глаза
    \item убивать время
    \item бисер метать
    \item волынку тянуть
    \item время истекло
    \item долгий ящик
    \item зарубить на носу
    \item ...
\end{itemize}

\begin{figure}[ht!]
    \centering
    \includegraphics[width=\textwidth]{hor}
\end{figure}

\begin{flushright}
    [И всё-таки \emph{фразеологизмы} вещь хорошая!]
\end{flushright}

\begin{flushleft}\parskip1em
    Правила отбора от Бора.

    Парень с Курил скурил все сигареты в блоке, сидя сутками с утками в блоке общежития, и из-за этого теперь почти в агонии ехал в вагоне.

    Он думал полететь в Тулузу, закатывая последний шар партии в ту лузу.

    Born, born in 1970, was a cool men.

    Смог смог помешать движению в городе. (\emph{ну или просто}) Смог который смог.

    You may shelter in our office off ice time (вы можете согреться в здании нашего офиса в холодные часы)

    Mess-age (испорченный возраст) it's time when a message (соц.сети) it's main in the life of teenagers.

    Вопрос: \emph{что означает Б. в имени Бенуа Б. Мандельброт?}\\
    Ответ: \emph{Бенуа Б. Мандельброт.}

    Подрубрика "Помощь молодому писателю" (начало какой-нибудь повести): набор заготовок для романов/дедективов/триллеров/...

    Распродажа уцененных персонажей 1-го и 2-го плана, а также 80\% скидка на залежавшиеся финалы для комедий.

    Германия. Герман и я, выйдя из аэропорта в это хмурое утро, оказались перед забором, который, в свою очередь, располагался за бором...

    \emph{4-смысленная фраза:} Хватит мять булки!

    Де Бройля всю жизнь волновали элементарные частицы.

    Штирлиц подсыпал яд врагу в рагу.

    Рентген любил просвещать людей.

    КОТЭ --- классно обманул товарища экзаменатора.

    \emph{--- Ошибка, которая привела к проигрышу партии!\\
    --- Да, в 91-м году...}
    \vspace*{-1em}\begin{flushright}
        Из разговоров за бильярдным столом, 11 сентября 2016
    \end{flushright}

    Та волга потерялась между полем, где росла таволга, и крутым берегом Волги. % так себе, но пойдёт

    \emph{Мама, я больше не Будда!}
\end{flushleft}
% реклама
\begin{figure}[ht!]
    \centering
    \includegraphics[width=\textwidth]{tea}
    \caption{Чай по-студенчески: без сахара и без заварки}
\end{figure}

\subsection{Recursive acronym}
\begin{flushleft}\parskip1em
    Рекурсивный акроним --- бэкроним (аббревиатура или акроним), который косвенно или напрямую ссылается на себя.\\
    Классика:\\
    ЛОМ --- лом обыкновенный металлический.\\
    GNU --- GNU's Not UNIX.

    \emph{\anttf{немного кошатинки на разминку:}}\\
    КОТ --- кот обманет тебя\\
    КОТЭ --- котэ обманул товарища экзаменатора\\
    ХВАТКА --- ХВАтит мяТь булКи! А!\\
    КРУТО --- круто разработал усовершенствование текущего опуса

    \emph{\anttf{и щепотка наркомании от Лёхи:}}\\
    ДЕЙСТВУЙ --- давай енту йохану скорее... также вувузелу у Йорика.\\
    ТИРЕ --- так и рождаются еноты.

    \emph{так и рождаются диалекты}\\
    STAR --- star to a rise\\
    \emph{если перевести rise как рассвет}
\end{flushleft}
% реклама
\begin{figure}[ht!]
    \centering
    \includegraphics[width=\textwidth]{cat}
    \caption{Всего за \$ 1.99}
\end{figure}

\subsection{Порядочно разные предложения}
Идея: изменять смысл предложения изменяя порядок слов.\\

Сlassic (for example): I would not —- Would I not


Я не хочу этого —- Хочу не я этого —— Не этого хочу я —— Я хочу не этого —--Хочу этого я? Неееее...


Реклама дверей: обнаружься! % висит над подземным переходом по ул. Мира около Комсы

\section{Минутка дзен}
{\color{white}Почему именно \author? Мы не знаем) Просто так получилось. Так так это минутка дзена, то за минуту можно прочитать от 120 до 180 символов, т.е. в среднем где-то 150 символов в минуту. Средняя длина слова в русском языке где-то 5.28 и поэтому здесь должен быть текст примерно на 750 символов. Встречайте текст: Подвес, по определению, неверифицируемо заставляет иначе взглянуть на то, что такое полином, при этом буквы А, В, I, О символизируют соответственно общеутвердительное, общеотрицательное, частноутвердительное и частноотрицательное суждения. Закон внешнего мира, как следует из полевых и лабораторных наблюдений, осмысленно транспонирует критерий интегрируемости. Плазменное образование выталкивает курс. Точность курса программирует дедуктивный метод.}

\vspace*{\fill}
\begin{center}
    \large 
    Спасибо за внимание!

    \vspace{2em}
    Есть идеи?
    \href{mailto:anto-kha0@rambler.ru}{Пишите нам}
\end{center}
\vfill
\thispagestyle{empty}
\end{document}
